\Recet{Cake à la banane}{Préparation : 20 minutes \\ Cuisson : 1 heure}
{}{\begin{itemize}
	\item 250 g de farine
	\item 140 g de sucre
	\item 2 cuillères à café de levure chimique
	\item une bonne pincée de sel
	\item 3 bananes moyennes mûres
	\item 85 g de beurre
	\item 2 cuillères à soupe de lait
	\item 2 oeufs
\end{itemize}}
{\phantom{.}

\medskip
\begin{enumerate}
	\item Ecraser une banane à la fourchette dans un bol et passer les deux autres au mixeur.
	\item Préchauffer le four à 165° C (thermostat 5-6).
	\item Mélanger 150 g de la farine avec le sucre, la levure chimique et le sel.
	\item Ajouter les bananes, ainsi que le beurre et le lait.
	\item Battre jusqu'à l'obtention d'une pâte homogène.
	\item Ajouter les oeufs et le reste de la farine, et bien mélanger.
	\item Graisser le fond d'une moule à cake, et y verser la pâte. Faire cuire au four à 165°C pendent 55 à 60 min (vérifier la cuisson).
\end{enumerate}

\medskip
\phantom{.}}

\bigskip
\Recet{Carrot cake}{Préparation : 30 minutes \\ Cuisson : 1 heure}
{}{\begin{itemize}
	\item 4 oeufs
	\item 175 g de farine
	\item 1 demi cuillerée à café de sel
	\item 300 g de sucre semoule
	\item 1 sachet de levure chimique
	\item 150 g de noix
	\item 200 g de carottes
	\item 10 cl d'huile de tournsesol
	\item 1 cuillerée à café de cannelle
	\item 50g de poudre d'amande
	\item 50g de poudre de noisettes
\end{itemize}}
{\phantom{.}

\medskip
\begin{enumerate}
	\item Préchauffer le four à 180°C (th 6)
	\item Eplucher puis les passer dans un mixer
	\item Mélanger la levure, la farine, le sel, la poudre d'amande et de noisettes et les différentes épices
	\item Battre les 4 oeufs et le sucre semoule dans un saladier
	\item Faire mousser le mélange
	\item Rajouter 2 cuillerées à soupe d'eau chaude
	\item Ajouter l'huile
	\item En plusieurs fois, incorporer la farine mélangée avec les épices et les carottes
	\item Beurrer un moule
	\item Verser la préparation dans le moule
	\item Faire cuire pendant 55 minutes (il faut que le gâteau soit sec)
\end{enumerate}

\medskip
\phantom{.}}

\bigskip
\Recet{Pumpkin bread}{Préparation : 30 minutes \\ Cuisson : 1 heure 30}
{Pour un bon pain}{\begin{itemize}
	\item 220 g de farine 
	\item 80 ml d'eau 
	\item 2 oeufs
	\item 300 g de sucre 
	\item 1/2 cuillère à café de levure
	\item 1/2 cuillère à café de levure alsacienne
	\item 1/2 cuillère à café de noix de muscade
	\item 1/2 cuillère à café de cannelle
	\item 120 ml d'huile neutre
	\item 245 g de purée de potiron
	\item 1 cuillère à café de sel
\end{itemize}}
{\phantom{.}

\bigskip
\begin{enumerate}
	\item Préchauffer le four à 175 degrés C 
	\item Dans un bol, mélanger la farine, les levures, le sel, la cannelle, le 4 épices, la noix de muscade et les clous de girofle
	\item Dans un autre bol, battre au batteur le sucre, les oeufs et l'huile
	\item Verser petit à petit la purée de citrouille tout en battant
	\item Rajouter en battant le mélange de farine
	\item Verser la pâte dans le moule
	\item Faire cuire 90 minutes
	\item Laisser refroidir 10 minutes avant de démouler
\end{enumerate}

\bigskip
\phantom{.}}

\bigskip
\Recet{Gâteau roulé}{Préparation : 20 minutes \\ Cuisson : 10 minutes}
{Pour 6 personnes}{\begin{itemize}
	\item 4 \oe ufs
	\item 125g de farine
	\item 100g de sucre
	\item 1 cuillerée à café de levure chimique (à voir : remplacer par du bicarbonate)
	\item 1 cuillère à café de fleur d'oranger.
	\item Garniture au choix (Gelée, confiture, crème chantilly avec fruits...)
\end{itemize}}
{\begin{enumerate}
	\item Allumer le four thermostat 7 / 210°C
	\item Séparez les blancs des jaunes. Travaillez les jaunes avec le sucre jusqu'à ce que le mélange devienne blanc. Ajoutez la farine petit à petit puis la levure.
	\item Battez les blancs en neige et incorporez-les au mélange. Ajouter la fleur d'oranger. Ne pas mélanger trop longtemps ou la pâte se liquéfiera.
	\item Beurrez un moule rectangulaire et y verser la pâte. Faire cuire à four chaud pendant 8 minutes.
	\item Démoulez le gâteau, laissez-le se refroidir un peu (surtout si vous y rajouter une crème du type crème au beurre ! Avec la chaleur, la crême coulera et le résultat sera d'autant moins appréciable. Dans ce cas, attendez une heure avec un torchon humide au-dessus pour qu'il ne durcisse pas). 
	\item Étalez la garniture et roulez le gâteau avec précaution. 
	\item Faîtes une décoration (le sucre glace, c'est pas mal, mais à faire à froid, ou le sucre se dissolvera !), et dégustez !
\end{enumerate}}

\bigskip
\Recet{Bûche à la crème au mascarpone}{}
{Pour 6 personnes}{Pour le biscuit :
\begin{itemize}
	\item 4 \oe ufs
	\item 125g de farine
	\item 100g de sucre
	\item 1 cuillerée à café de levure chimique (à voir : remplacer par du bicarbonate)
	\item 1 cuillère à café de fleur d'oranger.
	\item (optionnel) 2 cuillères à soupe de grand marnier (ou de cognac, ou de whisky)
\end{itemize}
Pour la crème :	
\begin{itemize}
	\item 300g de mascarpone
	\item Un goût au choix : ici, une tasse bien serrée de café, 50g de chocolat
	\item Au besoin : une feuille de gelatine
\end{itemize}}
{\phantom{.}

\bigskip
\phantom{.}
\begin{enumerate}
	\item Préparer un gâteau roûlé (cf. recette précédente)
	\item Dans une casserole, verser 2 cuillères à soupe de sucre et le Grand Marnier. Ajoutez le café et le chocolat, puis mettre le tout sur le feu. Faîtes bouillir et laisser sur le feu pendant 3 minutes, sans cesser de remuer.
	\item Napper le gâteau du sirop.
	\item Faîtes fondre la feuille de gelatine, et ajoutez la au mascarpone. Rajouter en même temps trois cuillères à soupe de sirop, et mélangez le tout.
	\item Etalez la garniture et roulez le gâteau avec précaution. Etalez le reste de garniture sur le bord, et faire des traits à la fourchette, et pensez à un peu de déco !
\end{enumerate}

\bigskip
\phantom{.}}

\bigskip
\Recet{Tarte aux noix au caramel}{Préparation : 30 minutes \\ Cuisson : 30 minutes}
{Pour 8 personnes}{Pour la pâte sâblée :
\begin{itemize}
	\item 220 g de farine 
	\item 20g de poudre de noisettes
	\item 125 g de beurre 
	\item 70 g de sucre semoule ou glace 
	\item 1 oeuf 
	\item 5 cl d'eau (ou de lait) 
	\item 1 pincée de sel 
\end{itemize}
Pour la tarte :
\begin{itemize}
	\item 250 g de cerneaux de noix
	\item 80 g de sucre
	\item 1 oeuf
\end{itemize}
Pour le caramel :
\begin{itemize}
	\item 100 g de sucre
	\item 100 g de crème fraîche(10 cl)
	\item 1/2 verre d'eau
\end{itemize}}
{\begin{enumerate}
	\item Préparer la pâte sablée : fouettez l'\oe uf avec le sucre et détendre avec un peu de lait
	\item Mettre le beurre en parcelles et le sel sur la farine et la poudre de noisettes.
	\item Écraser beurre et farine ensemble en frottant légèrement les mains. La farine passe entre les doigts.
	\item Le sablage terminé formez une fontaine. Verser au centre le mélange sucre + oeufs + eau.
	\item Serrer doucement la masse entre les mains. Farinez légèrement le plan de travail et fraiser jusqu'a obtention de l'amalgame. Former une boule.
	\item Préchauffer le four à 180°C. Étalez une pâte sur du papier sulfurisé, et la mettre dans un moule. Recouvrir le fond de pâte d'un rond de papier sulfurisé et remplir de haricots secs ou de noyaux de fruits. 
	\item Glisser la pâte au four et faire cuire 20 minutes. Mixer 50 g de noix avec 80 g de sucre et l'oeuf battu. Verser sur la pâte débarassée des haricots et du rond de papier sulfurisé. Mettre au four à 150°C (thermostat 5) une dizaine de minutes environ.
	\item Préparer le caramel mou : chauffer 100 g de sucre avec l'eau. Quand il caramélise, ajouter la crème et fouetter très fort quelques minutes.
	\item Garnir la tarte avec les cerneaux de noix. Les napper de caramel mou tiède. Laisser refroidir avant de servir.
\end{enumerate}}

\medskip
\phantom{.}

\bigskip
\Recet{Tarte à la rhubarbe et à la poudre d'amandes}{Préparation : 20 minutes \\ Cuisson : 40 minutes}
{Pour 6 personnes}{\begin{itemize}
	\item 500 g de Rhubarbe 
	\item 400 g de pâte sablée
	\item 100 g de beurre
	\item 100 g de sucre en poudre
	\item 100 g de cassonade
	\item 125 g de poudre d'amandes
	\item 2 oeufs
	\item 2 c. à soupe de farine
	\item 1 c. à café rase de cannelle en poudre
	\item 1 pincée de gingembre moulu
\end{itemize}}
{\begin{enumerate}
	\item Epluchez et coupez la rhubarbe en tronçons.
	\item Plongez-les une minute dans l’eau bouillante, puis égouttez-les.
	\item Mélangez 75 g de cassonade et les épices.
	\item Mettez la rhubarbe dans une casserole et saupoudrez-la du mélange sucre/épices.
	\item Placez la casserole sur feu doux et faites cuire 5 min en remuant.
	\item Retirez du feu, et laissez tiédir.
	\item Préchauffez le four th.6 (180°C).
	\item Mettez le beurre, le sucre, la poudre d’amandes, la farine et les œufs entiers dans le bol du mixeur.
	\item Faites tournez très rapidement jusqu’à obtention d’une crème.
	\item Etalez la pâte sur un plan de travail fariné et garnissez-en un moule à tarte beurré. Piquez le fond avec une fourchette et placez au frais.
	\item Mélangez la rhubarbe et la crème d’amande.
	\item Versez la préparation dans le fond de tarte.
	\item Enfournez et faites cuire 35 min.
	\item 10 min avant la fin de la cuisson, saupoudrez la tarte du reste de cassonade.
	\item Sortez la tarte du four et laissez refroidir avant de servir.
\end{enumerate}}
