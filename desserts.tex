\Recet{Cake à la banane}{Préparation : 20 minutes \\ Cuisson : 1 heure}
{}{\begin{itemize}
	\item 250 g de farine
	\item 140 g de sucre
	\item 2 cuillères à café de levure chimique
	\item une bonne pincée de sel
	\item 3 bananes moyennes mûres
	\item 85 g de beurre
	\item 2 cuillères à soupe de lait
	\item 2 oeufs
\end{itemize}}
{\phantom{.}

\medskip
\begin{enumerate}
	\item Ecraser une banane à la fourchette dans un bol et passer les deux autres au mixeur.
	\item Préchauffer le four à 165° C (thermostat 5-6).
	\item Mélanger 150 g de la farine avec le sucre, la levure chimique et le sel.
	\item Ajouter les bananes, ainsi que le beurre et le lait.
	\item Battre jusqu'à l'obtention d'une pâte homogène.
	\item Ajouter les oeufs et le reste de la farine, et bien mélanger.
	\item Graisser le fond d'une moule à cake, et y verser la pâte. Faire cuire au four à 165°C pendent 55 à 60 min (vérifier la cuisson).
\end{enumerate}

\medskip
\phantom{.}}

\bigskip
\Recet{Carrot cake}{Préparation : 30 minutes \\ Cuisson : 1 heure}
{}{\begin{itemize}
	\item 4 oeufs
	\item 175 g de farine
	\item 1 demi cuillerée à café de sel
	\item 300 g de sucre semoule
	\item 1 sachet de levure chimique
	\item 150 g de noix
	\item 200 g de carottes
	\item 10 cl d'huile de tournsesol
	\item 1 cuillerée à café de cannelle
	\item 50g de poudre d'amande
	\item 50g de poudre de noisettes
\end{itemize}}
{\phantom{.}

\medskip
\begin{enumerate}
	\item Préchauffer le four à 180°C (th 6)
	\item Eplucher puis les passer dans un mixer
	\item Mélanger la levure, la farine, le sel, la poudre d'amande et de noisettes et les différentes épices
	\item Battre les 4 oeufs et le sucre semoule dans un saladier
	\item Faire mousser le mélange
	\item Rajouter 2 cuillerées à soupe d'eau chaude
	\item Ajouter l'huile
	\item En plusieurs fois, incorporer la farine mélangée avec les épices et les carottes
	\item Beurrer un moule
	\item Verser la préparation dans le moule
	\item Faire cuire pendant 55 minutes (il faut que le gâteau soit sec)
\end{enumerate}

\medskip
\phantom{.}}

\bigskip
\Recet{Pumpkin bread (Pain à la cirtouille)}{Préparation : 30 minutes \\ Cuisson : 1 heure 30}
{Pour un bon pain}{\begin{itemize}
	\item 220 g de farine 
	\item 80 ml d'eau 
	\item 2 oeufs
	\item 300 g de sucre 
	\item 1/2 cuillère à café de levure
	\item 1/2 cuillère à café de levure alsacienne
	\item 1/2 cuillère à café de noix de muscade
	\item 1/2 cuillère à café de cannelle
	\item 120 ml d'huile neutre
	\item 245 g de purée de potiron
	\item 1 cuillère à café de sel
\end{itemize}}
{\phantom{.}

\bigskip
\begin{enumerate}
	\item Préchauffer le four à 175 degrés C 
	\item Dans un bol, mélanger la farine, les levures, le sel, la cannelle, le 4 épices, la noix de muscade et les clous de girofle
	\item Dans un autre bol, battre au batteur le sucre, les oeufs et l'huile
	\item Verser petit à petit la purée de citrouille tout en battant
	\item Rajouter en battant le mélange de farine
	\item Verser la pâte dans le moule
	\item Faire cuire 90 minutes
	\item Laisser refroidir 10 minutes avant de démouler
\end{enumerate}

\bigskip
\phantom{.}}

\bigskip
\Recet{Gâteau roulé}{Préparation : 20 minutes \\ Cuisson : 10 minutes}
{Pour 6 personnes}{\begin{itemize}
	\item 4 \oe ufs
	\item 125g de farine
	\item 100g de sucre
	\item 1 cuillerée à café de levure chimique (à voir : remplacer par du bicarbonate)
	\item 1 cuillère à café de fleur d'oranger.
	\item Garniture au choix (Gelée, confiture, crème chantilly avec fruits...)
\end{itemize}}
{\begin{enumerate}
	\item Allumer le four thermostat 7 / 210°C
	\item Séparez les blancs des jaunes. Travaillez les jaunes avec le sucre jusqu'à ce que le mélange devienne blanc. Ajoutez la farine petit à petit puis la levure.
	\item Battez les blancs en neige et incorporez-les au mélange. Ajouter la fleur d'oranger. Ne pas mélanger trop longtemps ou la pâte se liquéfiera.
	\item Beurrez un moule rectangulaire et y verser la pâte. Faire cuire à four chaud pendant 8 minutes.
	\item Démoulez le gâteau, laissez-le se refroidir un peu (surtout si vous y rajouter une crème du type crème au beurre ! Avec la chaleur, la crême coulera et le résultat sera d'autant moins appréciable. Dans ce cas, attendez une heure avec un torchon humide au-dessus pour qu'il ne durcisse pas). 
	\item Étalez la garniture et roulez le gâteau avec précaution. 
	\item Faîtes une décoration (le sucre glace, c'est pas mal, mais à faire à froid, ou le sucre se dissolvera !), et dégustez !
\end{enumerate}}

\bigskip
\Recet{Bûche à la crème au mascarpone}{}
{Pour 6 personnes}{Pour le biscuit :
\begin{itemize}
	\item 4 \oe ufs
	\item 125g de farine
	\item 100g de sucre
	\item 1 cuillerée à café de levure chimique (à voir : remplacer par du bicarbonate)
	\item 1 cuillère à café de fleur d'oranger.
	\item (optionnel) 2 cuillères à soupe de grand marnier (ou de cognac, ou de whisky)
\end{itemize}
Pour la crème :	
\begin{itemize}
	\item 300g de mascarpone
	\item Un goût au choix : ici, une tasse bien serrée de café, 50g de chocolat
	\item Au besoin : une feuille de gelatine
\end{itemize}}
{\phantom{.}

\bigskip
\phantom{.}
\begin{enumerate}
	\item Préparer un gâteau roûlé (cf. recette précédente)
	\item Dans une casserole, verser 2 cuillères à soupe de sucre et le Grand Marnier. Ajoutez le café et le chocolat, puis mettre le tout sur le feu. Faîtes bouillir et laisser sur le feu pendant 3 minutes, sans cesser de remuer.
	\item Napper le gâteau du sirop.
	\item Faîtes fondre la feuille de gelatine, et ajoutez la au mascarpone. Rajouter en même temps trois cuillères à soupe de sirop, et mélangez le tout.
	\item Etalez la garniture et roulez le gâteau avec précaution. Etalez le reste de garniture sur le bord, et faire des traits à la fourchette, et pensez à un peu de déco !
\end{enumerate}

\bigskip
\phantom{.}}

\bigskip
\Recet{Tarte aux noix au caramel}{Préparation : 30 minutes \\ Cuisson : 30 minutes}
{Pour 8 personnes}{Pour la pâte sâblée :
\begin{itemize}
	\item 220 g de farine 
	\item 20g de poudre de noisettes
	\item 125 g de beurre 
	\item 70 g de sucre semoule ou glace 
	\item 1 oeuf 
	\item 5 cl d'eau (ou de lait) 
	\item 1 pincée de sel 
\end{itemize}
Pour la tarte :
\begin{itemize}
	\item 250 g de cerneaux de noix
	\item 80 g de sucre
	\item 1 oeuf
\end{itemize}
Pour le caramel :
\begin{itemize}
	\item 100 g de sucre
	\item 100 g de crème fraîche(10 cl)
	\item 1/2 verre d'eau
\end{itemize}}
{\begin{enumerate}
	\item Préparer la pâte sablée : fouettez l'\oe uf avec le sucre et détendre avec un peu de lait
	\item Mettre le beurre en parcelles et le sel sur la farine et la poudre de noisettes.
	\item Écraser beurre et farine ensemble en frottant légèrement les mains. La farine passe entre les doigts.
	\item Le sablage terminé formez une fontaine. Verser au centre le mélange sucre + oeufs + eau.
	\item Serrer doucement la masse entre les mains. Farinez légèrement le plan de travail et fraiser jusqu'a obtention de l'amalgame. Former une boule.
	\item Préchauffer le four à 180°C. Étalez la pâte sur du papier sulfurisé, et la mettre dans un moule. Recouvrir le fond de pâte d'un rond de papier sulfurisé et remplir de haricots secs ou de noyaux de fruits. 
	\item Glisser la pâte au four et faire cuire 20 minutes. Mixer 50 g de noix avec 80 g de sucre et l'oeuf battu. Verser sur la pâte débarassée des haricots et du rond de papier sulfurisé. Mettre au four à 150°C (thermostat 5) une dizaine de minutes environ.
	\item Préparer le caramel mou : chauffer 100 g de sucre avec l'eau. Quand il caramélise, ajouter la crème et fouetter très fort quelques minutes.
	\item Garnir la tarte avec les cerneaux de noix. Les napper de caramel mou tiède. Laisser refroidir avant de servir.
\end{enumerate}}

\medskip
\phantom{.}

\bigskip
\label{TarteRhubarbe}
\Recet{Tarte à la rhubarbe et à la poudre d'amandes}{Préparation : 20 minutes \\ Cuisson : 40 minutes}
{Pour 6 personnes}{\begin{itemize}
	\item 500 g de Rhubarbe 
	\item 400 g de pâte sablée
	\item 100 g de beurre
	\item 100 g de sucre en poudre
	\item 100 g de cassonade
	\item 125 g de poudre d'amandes
	\item 2 oeufs
	\item 2 c. à soupe de farine
	\item 1 c. à café rase de cannelle en poudre
	\item 1 pincée de gingembre moulu
\end{itemize}}
{\begin{enumerate}
	\item Epluchez et coupez la rhubarbe en tronçons.
	\item Plongez-les une minute dans l’eau bouillante, puis égouttez-les.
	\item Mélangez 75 g de cassonade et les épices.
	\item Mettez la rhubarbe dans une casserole et saupoudrez-la du mélange sucre/épices.
	\item Placez la casserole sur feu doux et faites cuire 5 min en remuant.
	\item Retirez du feu, et laissez tiédir.
	\item Préchauffez le four th.6 (180°C).
	\item Mettez le beurre, le sucre, la poudre d’amandes, la farine et les œufs entiers dans le bol du mixeur.
	\item Faites tournez très rapidement jusqu’à obtention d’une crème.
	\item Etalez la pâte sur un plan de travail fariné et garnissez-en un moule à tarte beurré. Piquez le fond avec une fourchette et placez au frais.
	\item Mélangez la rhubarbe et la crème d’amande.
	\item Versez la préparation dans le fond de tarte.
	\item Enfournez et faites cuire 35 min.
	\item 10 min avant la fin de la cuisson, saupoudrez la tarte du reste de cassonade.
	\item Sortez la tarte du four et laissez refroidir avant de servir.
\end{enumerate}}

\medskip
\phantom{.}

\bigskip
\Recet{Trianon}{Préparation : 30 minutes \\ Repos : minimum 3h}
{Pour 12 personnes}{Pour la dacquoise à la noisette :
\begin{itemize}
	\item 90 g de poudre de noisettes
	\item 90 g de sucre glace
	\item 40 g de sucre en poudre
	\item 4 blancs d'oeufs battus en neige
\end{itemize}
Pour le feuilleté praliné :
\begin{itemize}
	\item 100 g de chocolat noir
	\item 100 g de chocolat au pralin
	\item 40 g de pralin
	\item 18 crêpes dentelles brisées
	\item 10 cl de crème liquide
\end{itemize}
Pour la mousse au chocolat :
\begin{itemize}
	\item 360 g de chocolat noir (de couverture si possible)
	\item 50 cl de crème liquide entière montée en chantilly
	\item 10 cl de lait
\end{itemize}}
{\phantom{.}

\medskip
\begin{enumerate}
	\item Pensez de suite à mettre un saladier et les fouets d'un batteur au congélateur pour la crème fouettée
	\item Dacquoise à la noisette : dans un saladier, mélangez la poudre de noisette, le sucre glace e le sucre en poudre. Dans un second saladier, battez les blancs en neige puis incorporez-y les poudres peu à peu tout en soulevant la masse délicatement.\\
Versez cette pâte en fine couche dans un cercle à pâtisserie de 26 à 28 cm de diamètre (ou un rectangle de 20$\times$23 cm environ ou dans un moule à fond amovible) et faites cuire 10 à 15 minutes à 170°c ou 180°c (thermostat 6). Surveillez le temps de cuisson. 
	\item Feuilleté praliné : faites fondre les 200g de chocolats, laissez refroidir un peu et ajoutez-y la crème fraîche, le pralin et les crêpes dentelles brisées. Mélangez et étalez cette pâte sur la dacquoise refroidie. 
	\item Mousse au chocolat : faites fondre le chocolat. \\
Montez la crème liquide en chantilly. Mélangez le chocolat avec le lait, puis ajoutez le mélange à la crème fouetée. Fouettez bien le tout et versez la mousse sur le feuilleté praliné.
	\item Laisser cet entremet au frais pour la nuit ou pour la journée puis décorez-le avec du cacao amer en poudre. Pensez à le sortir 1/2 heure à 1 heure avant la dégustation.
\end{enumerate}

\medskip
\phantom{.}}

\Recet{Croissants}{}
{Pour 12 croissants}{\begin{itemize}
	\item 500 g de farine T45
	\item 20 g de levure fraiche de boulangerie (cube) ou 10 g de levure sèche
	\item 60 g de sucre
	\item 10 g de sel
	\item 200 ml de lait
	\item 2 œufs + 1 pour dorer
	\item 200 g de beurre doux
	\item 10 cl d'eau
	\item 100 g de sucre
	\item 2 cs d' eau de fleur d'oranger
\end{itemize}}
{\begin{enumerate}
	\item Mettre la farine dans une terrine et faire un puits ; émietter la levure. Ajouter le sucre et le sel (le sel ne doit pas toucher la levure directement). Ajouter ensuite le lait tiède et les \oe ufs. Attention : penser à ne pas mettre la totalité du lait d'emblée et à ajuster la quantité finale exacte à la capacité d'absorption de la farine (qui dépend de la qualité de la farine). Pétrir de manière à obtenir une pâte souple et homogène. (Cette opération est encore plus simple si on dispose d'un robot batteur sur socle). Mettre en boule et laisser reposer 1 h dans une ambiance tiède (ou dans un four fonction étuve).
	\item Quand la pâte a doublé de volume, la malaxer quelques instants pour la faire retomber puis la placer au réfrigérateur pendant 30 mn 
	\item Étaler la pâte sur un plan de travail fariné à l'aide d'un rouleau à pâtisserie fariné. Frapper le beurre dans du film alimentaire pour qu'il ait à peu près la même consistance que la pâte. Le déposer au milieu sur la pâte en l'étalant, sans en mettre sur les bords.
	\item Allonger la pâte obtenue au rouleau en faisant attention de ne pas laisser échapper le beurre pour former un grand rectangle assez long pour que l'on puisse replier en 4 : rabattre les 2 extrémités vers le centre, puis replier en 2. Tourner la pâte pour avoir la pliure à droite.
	\item Allonger à nouveau la pâte pour former un grand rectangle, recommencer l'opération et laisser reposer 30 mn au réfrigérateur pour raffermir car le beurre a tendance à fondre. Ne pas hésiter d'ailleurs à mettre la pâte au réfrigérateur au cours des pliages si on sent qu'elle est trop molle (cela dépend aussi du temps...)
	\item Étaler la pâte finement, régulièrement, selon un rectangle d'environ 40 $\times$ 50 cm pour la découper en 2 bandes d'environ 20 cm $\times$ 50 cm. Faire des triangles de 20 $\times$ 10 cm environ. Rouler chaque triangle en finissant par la pointe. Préparer 2 tôles beurrées ou recouvertes de papier cuisson. Y déposer les croissants(environ 9 par plaques) en les espaçant un peu. Laissez reposer 1h.
	\item Porter à ébullition l'eau, le sucre et la fleur d'oranger. Badigeonner les croissants avec ce sirop pour plus de brillance.
	\item Préchauffer le four puis enfourner et cuire 10/15 mn à 210° en surveillant la coloration. 
\end{enumerate}}

\Recet{Havreflarns}{}
{Pour une cinquantaine de galettes}{Pour les galettes :
\begin{itemize}
\item 200g de flocons d'avoine
\item 300g de sucre
\item 180g de matière grasse de cuisine
\item 70g de farine
\item 20g de noix de coco râpée (facultatif)
\item 1 ou 2 gouttes de vanille liquide
\item 60g d'\oe uf (1 très gros \oe uf)
\item 1 bonne pincée de sel
\item 1/4 de cuillerée à café de levure chimique
\item 1/4 de cuillerée à café de de bicarbonate de sodium
\end{itemize}
Pour les galettes au chocolat :
\begin{itemize}
\item 300g de chocolat noir ou au lait
\end{itemize}}
{\begin{enumerate}
\item Dans un récipient, placez les flocons d'avoine, le sucre, la farine, la noix de coco et le sel. Mélanger sommairement.
\item Faître fondre la matière grasse de cuisoon (à la casserole ou au micro-ondes). Attention aux éclaboussures ! \\
Versez-la dans le récipient puis ajoutez l'\oe uf et la vanille liquide.
\item Laissez refroidir quelques minutes (si la matière grasse était trop chaude) puis ajouter la levure chimique et le bicarbonate.
\item Mélangez parfaitement la pâte puis mettez-la au réfrigérateur pendant au moins 2h.
\item Après ce temps, déalisez des boulettes de pâte. Il veut mieux les peser pour avoir un résultat parfait : 8g est l'idéal !
\item Préchauffer le four à 180°C. Placez les boulettes sur une plaque à pâtisserie garnie de papier suflurisé en les espaçant suffisamment car elles vont s'étaler. Faîtes cuire chaque fournée 8 à 9 minutes. Les galettes doivent être bien dorées sur le pourtour et le centre un peu plus clair. Transvasez sur une grille avec une spatule le temps qu'elles refroidissent et qu'elles durcissent.
\item Placez les havreflarns dans une boîte hermétique pour bien les conserver : l'humidité est la pire ennemie de ces petits gâteaux !

\bigskip
Pour les doubles galettes au chocolat :
\item Faire fondre le chocolat au bain-marie ou au micro-ondes (sans ajouter de chocolat dedans !). Si vous faîtes cela au micro-ondes, mettez le chocolat à fondre 20 secondes par 20 secondes en mélangeant à chaque fois.\\
Prenez une galette et posez-la à plat dans le chocolat fondu. Enfoncez un peu avec une fourcette jusqu'à ce que le bord touche le chocolat. Puis reprenez la galette et retournez la sur une grille. Prenez ensuite une autre galette et posez-la sur celle couverte de chocolat. Laissez bien refroidir et replacez les galettes dans une boîte hermétique. Ils se conservent un mois sans problème s'ils sont à l'abri de l'air et de l'humidité !
\end{enumerate}}

\Recet{L'assassin}{Préparation : 40 minutes \\ Cuisson : 45 minutes \\ Repos : 1 nuit}
{Pour 4 à 6 personnes}{\begin{itemize}
	\item 250g de sucre
	\item 150g de beurre demi-sel
	\item 180g d'oeufs
	\item 10g de farine
	\item 125g de chocolat noir à 60\%
\end{itemize}}
{\begin{enumerate}
\item Mélangez les \oe ufs et la farine au fouet électrique pendant 5 minutes pour aérer et faire mousser le tout
\item Mettez le sucre dans une casserole avec assez d'eau pour l'imbiber (en gros 45g d'eau pour 250g de sucre). Il faut bien veiller à ne pas avoir de sucre sur le bord de la casserole ou vous risqueriez de cristalliser tout le sucre à la fin. Une fois seulement que le sirop bout, vous pourrez mélanger avec une cuillère, mais pas avant !\\
Faîtes un caramel bien ambré puis ajoutez le beurre froid petit à petit hors du feu. Il faut l'incorporer doucement car à chaque ajout de beurre, le caramel va refroidir un petit peu et intégrera l'eau contenue dans le beurre au lieu de l'évaporer. Cela va crépiter puis rapidement se calmer. Mélangez pour incorporer autant que possible le beurre au caramel.
\item Mettez le mélange farine-\oe ufs dans le bol d'un robot. Tout en fouettant sur la plus basse vitesse, ajoutez le caramel sans attendre en le versant en filet sur le mélange \oe ufs-farine. Il faut verser le caramel chaud pour éviter qu'il ne durcisse. On obtient alors une pâte très joliment colorée.
\item Faites fondre le chocolat, puis versez-le dans la pâte. Préparer un moule à fond amovible de 15cm de diamètre, en le chemisant de papier sulfurisé. Versez la pâte puis faîtes cuire à 145°C.
\item Pour voir s'il est cuit, il suffit de secouer très légèrement le moule pour voir la réaction de la pâte : celle-ci doit trembler très légèrement au centre sans pour autant être liquide, un peu comme une gelée.\\
Laissez-le refroidir, puis mette le au frais pendant une nuit avant de le démouler. Découpez de fines part, il est bien meilleur comme ça !
\end{enumerate}

Si on cherche à faire un peu plus grand, multipliez chacun des ingrédients par ces coefficients :
\begin{itemize}
\item Moule de 16cm : 1,14
\item Moule de 18cm : 1,44
\item Moule de 20cm : 1,77
\item Moule de 22cm : 2,15
\item Moule de 24cm : 2,56
\end{itemize}}

\bigskip
\Recet{Tarte aux fraises sur lit de pistache}{Préparation : 1h30 \\ Cuisson : 20 minutes \\ Réfrigiration : 3 à 12h \\ \scriptsize{(La crème pâtissière peut se faire la veille)}}
{Pour 6 personnes}{\begin{itemize}
\item 1 pâte sablée (voir la recette de la \hyperref[TarteRhubarbe]{tarte à la rhubarbe})
\item 250g de fraises
\end{itemize}
Pour la crème de pistache :
\begin{itemize}
\item 50g de beurre fondu
\item 50g de poudre de pistaches (ou de pistaches fraîches réduits en poudre)
\item 50g de sucre glace
\item 5g de fécule de maïs (Maïzena)
\item 30g d'\oe uf
\item 10g de kirsch
\end{itemize}
Pour la crème patissière :
\begin{itemize}
\item 250ml de lait demi-écrémé
\item 25g de beurre doux
\item 25g de fécule de maïs
\item 50g de jaunes d'\oe ufs
\item 60g de sucre
\item 1 pincée de vanille en poudre
\end{itemize}
}{\begin{enumerate}
\item Commencez par préparer la crème pâtissière. Le mieux étant de la faire la veille pour qu'elle soit bien froide au moment de son utilisation. Mettez les 50g de jaunes d'\oe ufs dans un récipient avec la vanille en poudre, la moitié du sucre et la fécule. Mélangez au fouet. Faîtes bouillir le lait avec l'autre moitié du sucre et le beurre.\\
Quand le lait bout, versez-le en filet sur le mélange jaunes-poudre à crème tout en fouettant. Mélangez bien puis reversez cette préparation dans la casserole. Remettez à cuire sur feu doux.\\
Laissez bouillir 3 à 4 minutes toujours en mélangeant vivement.\\
Versez la crème pâtissière dans un plat puis couvrez d'un film étirable directement au contact. Laissez refroidir complètement puis placez au réfrigérateur pour 3 heures ou pour une nuit entière.
\item Le jour même, étalez votre pâte sur 2 ou 3mm d'épaisseur puis foncez un cercle de 18cm. Mettez le tout au réfrigérateur. Préchaufez votre four à 180°C.
\item Préparez la crème de pistaches. \\
Dans un bol, mettez la poudre de pistache, le sucre glace, la fécule et le beurre mou. Ajoutez le kirsch et les 30g d'\oe uf, puis mélangez au fouet. Versez la crème d'amande dans le cercle.
\item Enfournez la tarte pour une vingtaine de minutes. Il faut adapter le temps de cuisson à ce que vous voyez ! La tarte doit être bien dorée dessus et dessous. Laissez refroidir sur une grille à pâtisserie.
\item Dans un grand bol, mettz la crème pâtissière bien froide puis fouettez la pendant 1 ou 2 minutes pour bien la lisser et l'assouplir.\\
Versez-la dans la tarte sur la crème de pistaches.
\item Coupez les fraises en deux en enlevant la queue puis placez-les sur la tarte. Mettez au frais jusqu'au moment de servir !
\end{enumerate}}

\medskip
\Recet{Far breton}{Préparation : 20 minutes \\ Cuisson : 60 minutes}
{Pour 6 à 8 personnes}{\begin{itemize}
\item 220 g de farine
\item 130 g de sucre en poudre
\item 1 sachet de sucre vanillé
\item 50cl de lait
\item 25cl de crème fraiche entière
\item 5 oeufs
\item 20 g de beurre
\item 500 g de pruneaux (Facultatif)
\end{itemize}}
{\begin{enumerate}
\item Pour commencer, préchauffez le four à 180°C (thermostat 6).
\item Dans un saladier, mélangez le sucre, la farine et ajoutez le sucre vanillé. Puis ajoutez les œufs en prenant soin de bien mélanger délicatement le tout à chaque fois. Versez le lait et ajoutez le beurre au préalablement fondu puis mélangez jusqu'à ce que vous obteniez une pâte homogène.
\item Ajoutez vos pruneaux si vous souhaitez obtenir un far aux pruneaux et pensez à les dénoyauter s'ils ont des noyaux (mais vous pouvez évidemment le déguster nature -ce qui est d'ailleurs plus traditionnel- ou avec des pommes \bcinfo).
\item Beurrez le fond de votre moule et versez y la pâte.
\item Vous pouvez placer votre moule au four et patientez une heure environ.
\end{enumerate}
\bcinfo\ Pour un far aux pommes, faîtes revenir à la poêle 500g de pommes coupées en morceaux dans 30g de beurre.}

\medskip
\Recet{Flan Parisien}{Préparation : 1h30 \\ Cuisson : 45 minutes \\ Réfrigération : 3h30}
{Pour un flan de 24cm}{Pour la pâte fécule \begin{itemize}
\item 250g de farine de type 55
\item 50g de fécule de pomme de terre
\item 225g de beurre
\item 55g de lait
\item 15g de jaune d'\oe uf
\item 5g de sel
\item 30g de sucre
\end{itemize}
Pour la crème à flan
\begin{itemize}
\item 180g de sucre
\item 130g de fécule de maïs
\item 2 goutte de vanille liquide
\item 160g d'\oe ufs
\item 60g de jaunes d'\oe ufs
\item 30g de beurre doux
\item 1,3l de lait demi-écrémé\\+135g de sucre semoule
\end{itemize}}
{\begin{enumerate}
\item Préparez la pâte décule : mettez le beurre froid coupé en morceaux, la farine, la fécule, le sel et le sucre dans un bol de robot mélangeur muni de la feuille. On peut bien sûr pétrir à la main. Mélangez jusqu'à obtenir une poudre sableuse.
\item Dans un petit bol, fouettez ensemble le jaune d'\oe uf et le lait. Versez sur le mélange sableux et remuez jusqu'à ce que la pâte soit homogène. La pâte va sans doute paraître un peu molle, surtout si vous faîtes cette recette un jour où il fait chaud. Mettez la pâte sous film étirable et stockez-la 30 minutes au frais.
\item Étalez la pâte sur un plan de travail légèrement fariné ou sur un papier sulfurisé sur une épaisseur de 2mm. S'il fait chaud, procédez par étapes ! Commencez à étaler et si vous voyez que vous avez du mal mettez la pâte au frais quelques minutes.
\item Foncez un cercle à entremet de 24cm de diamètre posé sur une plaque à pâtisserie garnie de papier sulfurisé (ou un moule à bords haut, mais le resultat sera un peu moins beau !). Assurrez-vous de bien plaquer la pâte sur le fond et sur le bord du cercle. Piquez le fond de pâte puis mettez au réfrégirétaeur jusqu'à utilisation.
\item Préchauffez le fout à 180°C
\item Préparez maintenant la crème à flan : mettez les 180g de sucre, la vanille et la fécule de maïs dans un récipient. Ajoutez les \oe ufs entiers et les jaunes. 
\item Portez le lait avec le beurre et les 135g de sucre à ébullition dans une grande casserole sur feu doux. Quand le lait bout, versez en filet sur le mélange précédeux tout en fouettant. Mélangez bien puis reversez le tout dans la casserole.
\\ Laissez bouillir sur feux doux 3 à 4 minutes toujours en mélangeant avec un fouet.Versez l'intégralité de la crème dans le cercle foncé. Lissez la surface de la crème avec une spatule puis en secouant légèrement la plaque.
\item Enfournez à 180°C pendant 40 à 50 minutes. Le flan doit ressortir bien gonflé et coloré. Laissez-le complètement refroidir, puis décerclez le soigneusement. Mettez le alors au frais pendant au moins 3 heures.
\end{enumerate}}

\medskip
\Recet{Tiramisu}{Préparation : 30 minutes \\ Réfrigération : 3h}
{Pour 6 personnes}{\begin{itemize}
\item 500g de mascarpone
\item 2 gros \oe ufs
\item 80g de sucre semoule
\item 30g de sucre glace
\item 2 paquets de biscuits à la cuiller
\item 1 tasse de café bien fort
\item 2 à 3 cuillères à soupe d'amaretto ou de kirsch (Facultatif)
\item Cacao en poudre (non sucré de préférence)
\end{itemize}}
{\begin{enumerate}
\item Commencez par séparer les jaunes des blancs. Mettez les jaunes avec  le sucre semoule et fouettez le tout avec un batteur électrique pour blanchir les jaunes le plus possible. Ajoutez le mascarpone et continuez de fouetter pour obtenir une crème aérée, une quasi-chantilly.
\item Dans un autre récipient, montez les blancs en neige bien ferme en ajoutant, à mi-parcours, le sucre glace. Ajoutez ces blancs à la crème au mascarpone. Mélangez délicatement avec une maruse en soulvant la crème pour ne pas faire retomber les blancs.
\item Préparez un café assez fort et ajoutez-y l'amaretto. Trempez un à un les bisctuis dans le café pendant 4 à 5 secondes pour bien les imbiber, mais sans les détremper ! Placez les bisctuis dans un plat de service de 18 sur 24cm en les serrant bien. Versez dessus la crème au mascarpone.
\item Laissez reposer au moins 3 heures au réfriférateur. Au dernier moment, saupoudrez de cacao avec un petit tamis. 
\end{enumerate}}

\medskip
\Recet{Millionaire shortbreads}{Préparation : 40 minutes \\ Cuisson : 40 minutes \\ Réfrigération : 2h}
{Pour un moule de 21 $\times$ 21cm}{Pour la pâte à shortbreads\begin{itemize}
\item 60g de sucre
\item 120g de beurre saké
\item 180g de farine
\end{itemize}
Pour le caramel \begin{itemize}
\item 150g de vergeoise blonde
\item 25g de miel
\item 175g de beurre
\item 1 boîte de lait concentré
\item 1/4 de cuillère à café de sel
\item 200g de chocolat au lait
\end{itemize}}
{\begin{enumerate}
\item Préchauffez votre four à 180°C
\item Mettez le beurre salé mou et le sucre dans un bol. Mélangez pour obtenir une sorte de crème (avec la feuille si vous avez un robot !). Ajoutez la farine. Mélangez et ramenez la pâte en boule sans travailler de façon excessive. Mettez la boule dans un moule de 21 $\times$ 21cm garni de papier sulfurisé. Étalez uniformément à la main puis avec le dos d'une cuillère à soupe pour avoir une pâte bien régulière. Mettez au four pendant 25 minutes.
\item Préparez le caramel en mettant la vergeoise, le sel, le miel et le beurre dans une casserole. Mettez sur feu doux et portez à ébullition. Laissez bouillir 3-4 minutes. Ajoutez ensuite la boîte de lait concentré sucré. Mélangez bien puis laissez cuire 8-10 minutes à partir de l'ébullition sur feu doux tout en mélangeant constamment.
\item Versez le caramel sur la pâte cuite, toujours dans le moule carré. Laissez refroidir totalement puis mettez 2 heures au frais.
\item Mettez le chocolat à fondre puis versez-le sur le caramel. Répartissez-le bien en remuant le moule pour avoir une surfasse lisse. Quand le chocolat commence à prendre mais qu'il n'est pas encore dur (10 minutes après le versement à peu près), coupez des parts avec une lame bien coupante trempée dans de l'eau chaude.
\item Conservez en boîte hermétique.
\end{enumerate}}

\Recet{Bugnes}{Préparation : 3h \\ Cuisson : 40 minutes}
{Pour une vingtaine de bugnes}{\begin{itemize}
\item 300 g de farine tamisée
\item 70 g de sucre en poudre
\item 80 g de beurre doux juste fondu mais pas chaud
\item 3 gros \oe ufs
\item 5 g de sel
\item eau de fleur d'oranger ou autre arôme
\end{itemize}}
{\begin{enumerate}
\item Mettre la farine, le sel, le sucre en fontaine. Ajouter les oeufs et mélanger
\item Ajouter le beurre, l'eau de fleur d'oranger et incorporer le tout dans la farine. Mélanger intimement.
\item Laisser reposer au moins deux heures en un endroit tempéré.
\item Diviser la pâte en deux pour abaisser avec plus d'aisance.
\item Abaisser et parer les bords avec une roulette cannelée ou au couteau. Ça donne des triangles de pâte fendus au milieu et dont on passe une des pointe à travers.
\item Et puis on passe en friture chaude, mais pas trop. 170/180°C c'est ce qu'on appelle une "petite friture". Très vite les bugnes gonflent et colorent, il faut les retourner pour obtenir une coloration uniforme.
\item La cuisson est terminée en quelques secondes. Attention à ne pas laisser noircir l'huile qui sera définitivement perdue après la cuisson.
\item On débarrasse rapidement sur une feuille de papier absorbant après avoir égoutté dans l'écumoire ou l'araignée et on saupoudre de sucre glace.
\end{enumerate}}
