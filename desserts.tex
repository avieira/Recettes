\Recet{Assassin}{Préparation : 40 minutes \\ Cuisson : 45 minutes \\ Repos : 1 nuit}
{Pour 4 à 6 personnes}{\begin{itemize}
	\item 250g de sucre
	\item 150g de beurre demi-sel
	\item 180g d'oeufs
	\item 10g de farine
	\item 125g de chocolat noir à 60\%
\end{itemize}}
{\begin{enumerate}
\item Mélangez les \oe ufs et la farine au fouet électrique pendant 5 minutes pour aérer et faire mousser le tout
\item Mettez le sucre dans une casserole avec assez d'eau pour l'imbiber (en gros 45g d'eau pour 250g de sucre). Il faut bien veiller à ne pas avoir de sucre sur le bord de la casserole ou vous risqueriez de cristalliser tout le sucre à la fin. Une fois seulement que le sirop bout, vous pourrez mélanger avec une cuillère, mais pas avant !\\
Faîtes un caramel bien ambré puis ajoutez le beurre froid petit à petit hors du feu. Il faut l'incorporer doucement car à chaque ajout de beurre, le caramel va refroidir un petit peu et intégrera l'eau contenue dans le beurre au lieu de l'évaporer. Cela va crépiter puis rapidement se calmer. Mélangez pour incorporer autant que possible le beurre au caramel.
\item Mettez le mélange farine-\oe ufs dans le bol d'un robot. Tout en fouettant sur la plus basse vitesse, ajoutez le caramel sans attendre en le versant en filet sur le mélange \oe ufs-farine. Il faut verser le caramel chaud pour éviter qu'il ne durcisse. On obtient alors une pâte très joliment colorée.
\item Faites fondre le chocolat, puis versez-le dans la pâte. Préparer un moule à fond amovible de 15cm de diamètre, en le chemisant de papier sulfurisé. Versez la pâte puis faîtes cuire à 145°C.
\item Pour voir s'il est cuit, il suffit de secouer très légèrement le moule pour voir la réaction de la pâte : celle-ci doit trembler très légèrement au centre sans pour autant être liquide, un peu comme une gelée.\\
Laissez-le refroidir, puis mette le au frais pendant une nuit avant de le démouler. Découpez de fines part, il est bien meilleur comme ça !
\end{enumerate}

Si on cherche à faire un peu plus grand, multipliez chacun des ingrédients par ces coefficients :
\begin{itemize}
\item Moule de 16cm : 1,14
\item Moule de 18cm : 1,44
\item Moule de 20cm : 1,77
\item Moule de 22cm : 2,15
\item Moule de 24cm : 2,56
\end{itemize}}

\bigskip
\Recet{Brioche}{}
{}{\begin{itemize}
	\item 250g de farine T55
	\item 3g de sel
	\item 1/2 cuillerée à café (rase) de levure de boulangerie sèche (ou 5g de levure fraiche)
	\item 50g de sucre
	\item 100g d'œuf (2 œufs moyens)
	\item 60ml de lait 
	\item 125g de beurre doux
\\
	\item 1 œuf
	\item 2 cuillerées à soupe de lait
\end{itemize}}
{\begin{enumerate}
	\item  Mettez la farine, le sel, le sucre et la levure sèche dans un bol. (Si vous prenez de la levure fraiche, diluez-la dans le lait légèrement tiédi). Ajoutez les 100g d'\oe uf, puis mélangez 10-15 minutes au crochet pétrisseur, 20 minutes à la main. En même temps, versez petit à petit le lait.
	\item Ajoutez le beurre à température ambiante. Si vous avez oublié de le sortir du réfrigérateur avant, pas de panique ! Vous pouvez le mettre froid coupé en petits morceaux, il mettra juste plus de temps à s'incorporer. Pétrissez de nouveau pendant 10 minutes au crochet, 15 minutes à la main. Il faut que la pâte devienne bien élastique et très souple. Le beurre doit bien sûr être incorporé et elle se détache un peu du bol.
	\item  Laissez gonfler 1h30 dans un endroit tiède, avec un film étirable directement au contact de la pâte. Vous pouvez la mettre à 40°C au four si vous avez cette fonction basse température. Elle doit doubler de volume. 
	\item Après ce temps, appuyez dessus pour enlever l'air (la pâte colle un peu c'est normal) qu'elle contient puis emballez-la bien de film étirable. Placez ce paquet au réfrigérateur pour la nuit. Avant de la mettre au frais, la pâte n'est pas encore ferme, mais très souple et un peu collante. Ne pensez pas que c'est raté ! C'est normal. Il faudra peut être y aller avec une maryse pour l'emballer de film étirable.
	\item  Le lendemain, déballez la boule de pâte froide et placez-la sur un plan de travail légèrement fariné. Cette fois, le beurre a refroidi et la pâte est plus facilement manipulable. Avec un voile de farine, elle ne collera plus. Appuyez dessus pour enlever l'air qu'elle contient. Pesez votre pâte: elle doit faire en gros dans les 580g. Il y a toujours des pertes entre les ustensiles, le film étirable etc. Pesez une boule de 90g et roulez-la avec un peu de farine, pas trop non plus ! Roulez le reste de pâte en une autre boule.
	\item Vaporisez de la bombe de démoulage dans votre moule à brioche. Vous pouvez aussi bien sûr y mettre du beurre. Placez la grosse boule de pâte dans le fond et appuyez au centre pour y faire une sorte de cavité. C'est en faisant cela que vous empêcherez la pâte de gonfler de travers. Avec une petite boule sur le dessus, la brioche tiendra en cuisson. Si vous mettez une boule trop grosse et sans cavité, la brioche poussera sur le côté et ressemblera à une vieille chaussette ! Elle sera quand même bonne ! Posez la petite boule au centre puis laissez gonfler pendant 1h à 40°C ou 1h30-2h à température ambiante. 
	\item  Pour la dorure, mélangez l'oeuf battu avec le lait puis passez-la sur la surface de la pâte délicatement au pinceau. Laissez sécher 10 minutes puis mettez une deuxième couche. Ne mettez pas trop de dorure sur votre pinceau, sinon elle coulera vers le moule: en cuisson, cela donnera des tâches brûlées. 
	\item  Enfournez et laissez cuire 25 minutes à 175°C: la pâte va gonfler encore plus et se dorer. Et voilà le résultat ! Une brioche à tête bien gonflée ! Elle sera délicieuse tiède, froide ou même grillée au grille-pain coupée en tranches avec de la confiture pour le petit déjeuner !
\end{enumerate}}

\bigskip
\Recet{Brownies Chocolat Bananes et Beurre de Cacahuètes}{}
{Pour un moule carré de 21x21cm (16 parts)}{\begin{itemize}
	\item 70g de beurre doux
	\item 70g de beurre demi-sel
	\item 210g de chocolat
	\item 2 gros œufs
	\item 165g de sucre
	\item 2 cuillerées à café de vanille liquide
	\item 45g de farine
	\item 1 banane
	\item 115g de chocolat coupé en "chunks" ou en pépites
	\item 110g de beurre de cacahuètes "crunchy"
\end{itemize}}
{\begin{enumerate}
	\item Mettre le chocolat et le beurre à fondre au bain-marie ou au micro-ondes. Dans un autre bol, mélanger les deux gros \oe ufs, le sucre et la vanille. Ne pas trop mélanger pour avoir une pâte molle et non aérée. 
	\item Verser le chocolat fondu, mais pas chaud pour ne pas cuire les \oe ufs. Ajouter la farine. Ajouter le chocolat coupé en morceaux et la moitié de la banane écrasée à la fourchette. Bien mélanger. Ajouter enfin la deuxième moitié de banane cette fois coupée en petits morceaux. 
	\item Verser dans un moule carré de 21X21 garni de papier sulfurisé. Bien étaler la pâte. Ajouter le beurre de cacahuètes en petits paquets (mieux vaut du crunchy !). Avec un couteau ou une baguette chinoise, répartir grossièrement le beurre de cacahuètes pour que chaque part finale en ait. 
	\item Mettre au four à 170°C pendant 30 minutes. Les côtés doivent commencer légèrement à se colorer et le reste doit paraitre cuit en surface mais pas trop en dessous ! Laisser refroidir complètement puis mettre au frais deux heures avant de démouler et de couper en parts. 
\end{enumerate}}

\bigskip
\Recet{Bûche à la crème au mascarpone aux deux chocolats}{}
{Pour 6 personnes}{Pour le biscuit :
\begin{itemize}
	\item 4 \oe ufs
	\item 125g de farine
	\item 100g de sucre
	\item 1 cuillerée à café de levure chimique (à voir : remplacer par du bicarbonate)
	\item 1 cuillère à café de fleur d'oranger.
	\item (optionnel) 2 cuillères à soupe de grand marnier (ou de cognac, ou de whisky)
\end{itemize}
Pour la crème :	
\begin{itemize}
	\item 20cl de kirch
	\item 20cl de sirop de canne à sucre
	\item 15g de cacao amer en poudre
	\item 500g de mascarpone
	\item 50cl de crème fleurette
	\item 1 sachet de chantifix (facultatif)
	\item 300g de chocolat noir
	\item 100g de chocolat blanc
	\item 15cl de crème liquide entière
\end{itemize}}
{\phantom{.}
\begin{enumerate}
	\item Avant toute chose, mettez dans un saladier le mascarpone, la crème fleurette et les batteurs. Mettez le tout au frigo au moins 30 minutes : si le tout est bien froid, la chantilly montera plus facilement !
	\item Préparez la génoise pour un gâteau roûlé (\hyperref[gateauRoule]{recette} p.\pageref{gateauRoule}).
	\item Quand le gâteau est encore chaud, mélangez le kirch, le sirop de canne à sucre et le cacao, puis ponchez le gâteau roulé. Laissez le gâteau refroidir sous un torchon humide.
	\item Dans deux bols différents, cassez en morceaux les deux chocolats (sans les mélanger !). Portez à ébullition les 15cl de crème liquide entière. puis mettre deux-tiers de la crème dans le chocolat noir, et le reste dans le chocolat blanc. Mélanger bien chacun des bols pour faire fondre les chocolats. Au besoin, vous pouvez remettre le bol 20 secondes au micro-ondes pour rechauffer un peu le tout et éliminer les derniers morceaux non fondus.
	\item Sortez le saladier du frigo et montez la chantilly. À mi-parcours, ajouter le sachet de chantifix. Normalement, si le tout était bien refroidi,la chantilly monte sans soucis !
	\item Dans un autre saladier, mettez le chocolat noir et deux tiers de votre chantilly. Mélanger en soulevant bien la pâte (comme pour une mousse au chocolat !). Une fois que le tout est bien mélangé, étalez le tout sur la génoise. Roulez le gâteau avec précaution.
	\item Faîtes de même pour mélanger le chocolat blanc avec la chantilly. Étalez le mélange sur le pourtour du gâteau, puis formez des traits avec une fourchette. Pous pouvez ensuite ajouter un peu de déco (pralin, vermicelles de chocolat...).
\end{enumerate}
\bcinfo\ Vous pouvez bien sûr remplacer les deux chocolas par autre chose. Dans ce cas, pensez aussi à enlever le cacao du sirop de ponchage ! 
\phantom{.}}

\bigskip
\Recet{Bugnes}{Préparation : 3h \\ Cuisson : 40 minutes}
{Pour une vingtaine de bugnes}{\begin{itemize}
\item 300 g de farine tamisée
\item 70 g de sucre en poudre
\item 80 g de beurre doux juste fondu mais pas chaud
\item 3 gros \oe ufs
\item 5 g de sel
\item eau de fleur d'oranger ou autre arôme
\end{itemize}}
{\begin{enumerate}
\item Mettre la farine, le sel, le sucre en fontaine. Ajouter les oeufs et mélanger
\item Ajouter le beurre, l'eau de fleur d'oranger et incorporer le tout dans la farine. Mélanger intimement.
\item Laisser reposer au moins deux heures en un endroit tempéré.
\item Diviser la pâte en deux pour abaisser avec plus d'aisance.
\item Abaisser et parer les bords avec une roulette cannelée ou au couteau. Ça donne des triangles de pâte fendus au milieu et dont on passe une des pointe à travers.
\item Et puis on passe en friture chaude, mais pas trop. 170/180°C c'est ce qu'on appelle une "petite friture". Très vite les bugnes gonflent et colorent, il faut les retourner pour obtenir une coloration uniforme.
\item La cuisson est terminée en quelques secondes. Attention à ne pas laisser noircir l'huile qui sera définitivement perdue après la cuisson.
\item On débarrasse rapidement sur une feuille de papier absorbant après avoir égoutté dans l'écumoire ou l'araignée et on saupoudre de sucre glace.
\end{enumerate}}

\bigskip
\Recet{Cailloux Amandes-Cannelle}{}
{}{\begin{itemize}
	\item 180g d'amandes
	\item 300g de sucre
	\item 1+1/2 cuillerée à café de cannelle
\end{itemize}}
{\begin{enumerate}
	\item Mettre 180 (sur les 300g) de sucre et une cuillerée à café de cannelle dans un large poêle et mélanger.  Ajouter 30 à 40g d'eau, juste de quoi imbiber le sucre. 
	\item Mettre sur feu modéré. Le mélange va bouillir. Ajouter les amandes, bien mélanger et cuire pour avoir de gros bouillons. Toujours mélanger avec une cuillère en bois.
	\item Le tout va sabler. C'est à dire que le sucre va cristalliser. Une fois que le sucre cristallise totalement, verser les amandes dans un récipient. 
	\item Ajouter le reste de sucre (normalement 120g si on suit bien!) et la demie cuillerée à café de cannelle.  Ajouter un peu d'eau et porter à gros bouillons.
	\item Quand on voit que le sucre est prêt  à recristalliser, ajouter les amandes. Bien enrober de sucre toutes les amandes. Il y a donc maintenant une couche plus épaisse de sucre et c'est ce que l'on souhaite. Le sucre va cristalliser de nouveau et sécher. 
	\item Verser dans un plat et laisser refroidir avant d'engloutir ! Conserver absolument en boîte hermétique !
\end{enumerate}}

\bigskip
\Recet{Cake au citron}{}
{}{\begin{itemize}
	\item 200g de sucre (il n'y a qu'avec cette quantité que le cake fonctionne et cela équilibre le citron)
	\item 120g de beurre fondu
	\item le zeste d'un citron jaune
	\item 165g d'œuf
	\item 150g de farine
	\item 80g de jus de citron
	\item 1 demi cuillerée à café de levure chimique
\end{itemize}
Glaçage:
\begin{itemize}
	\item 25g de jus de citron
	\item 130g de sucre glace
\end{itemize}}
{\begin{enumerate}
	\item Préchauffez votre four à 170°C.
	\item Faites fondre le beurre au four à micro-ondes ou au bain-marie. Versez le sucre dans un bol avec le beurre et le zeste d'un citron. Mélangez sommairement le beurre fondu avec le sucre. Ajoutez les 165g d'œuf, en gros 3 œufs moyens. Ajoutez la farine et la levure chimique. Et enfin le jus de citron.
	\item Versez la pâte dans un petit moule à cake de 160x80mm, légèrement beurré.
	\item Enfournez et laissez cuire 30 à 40 minutes. le gâteau doit être légèrement doré, pas trop gonflé. Vérifiez qu'il est cuit à l'intérieur en plantant un pique à brochette ou une lame de couteau qui devra ressortir sèche. Démoulez-le à l'envers sur une grille.
	\item Emballez-le immédiatement de film étirable pour qu'il conserve toute son humidité. De cette manière, il gardera sa texture moelleuse et ferme à la fois.
	\item Laissez le cake refroidir totalement dans son emballage. Quand il est à température ambiante, préparez le glaçage en mélangeant le sucre glace avec le jus de citron.
	\item Déballez le cake puis versez le glaçage. Mettez bien sûr une assiette en dessous pour récupérer l'excédent. 
	\item Remettez le cake sur sa grille au four à 100°C pendant 8 minutes pour sécher le glaçage. Si vous touchez le cake à ce moment, il doit être soyeux au toucher: le glaçage est bien sec et très doux. 
	\item Laissez-le refroidir puis dégustez en petites tranches...
\end{enumerate}}

\bigskip
\Recet{Cake à la banane}{Préparation : 20 minutes \\ Cuisson : 1 heure}
{}{\begin{itemize}
	\item 250 g de farine
	\item 140 g de sucre
	\item 2 cuillères à café de levure chimique
	\item une bonne pincée de sel
	\item 3 bananes moyennes mûres
	\item 85 g de beurre
	\item 2 cuillères à soupe de lait
	\item 2 oeufs
\end{itemize}}
{\phantom{.}

\medskip
\begin{enumerate}
	\item Ecraser une banane à la fourchette dans un bol et passer les deux autres au mixeur.
	\item Préchauffer le four à 165° C (thermostat 5-6).
	\item Mélanger 150 g de la farine avec le sucre, la levure chimique et le sel.
	\item Ajouter les bananes, ainsi que le beurre et le lait.
	\item Battre jusqu'à l'obtention d'une pâte homogène.
	\item Ajouter les oeufs et le reste de la farine, et bien mélanger.
	\item Graisser le fond d'une moule à cake, et y verser la pâte. Faire cuire au four à 165°C pendent 55 à 60 min (vérifier la cuisson).
\end{enumerate}

\medskip
\phantom{.}}

\bigskip
\Recet{Cake à la châtaigne}{}
{Pour un moule à cake de 18x7cm}{\begin{itemize}
	\item 120g de beurre doux
	\item 105g de sucre
	\item 145g d'œuf
	\item 225g de crème de marrons
	\item 80g de farine
	\item 60g de farine de châtaigne
	\item 1/2 paquet de levure chimique 
\end{itemize}}
{\phantom{.}

\medskip
\begin{enumerate}
	\item Faites fondre le beurre au four à micro-ondes ou au bain-marie. 
	\item Dans un récipient, mélangez le beurre fondu, le sucre et les \oe ufs.  Ajoutez la crème de marrons.  Puis terminez avec la farine, la farine de châtaigne et la levure chimique.
	\item Versez la pâte dans le moule à cake chemisé de papier sulfurisé.
	\item Faites cuire à 180°C pendant 35 à 45 minutes. Surveillez bien la cuisson et adaptez-la à votre propre four. La lame d'un couteau doit en ressortir bien sèche ! Le cake va gonfler tout en restant assez plat.
	\item  Démoulez le cake sur une grille à pâtisserie puis laissez-le refroidir. Conservez-le dans du film étirable. 
\end{enumerate}

\medskip
\phantom{.}}

\bigskip
\Recet{Cake rose litchi}{}
{}{\begin{itemize}
	\item 200g de litchis (pesé sans la peau et les noyaux)
	\item 150g de sucre semoule
	\item 150g d'\oe ufs
	\item 100g de poudre d'amande
	\item 250g de farine
	\item 8g de levure chimique
	\item 150g de beurre doux
	\item 3 cuillères à soupe d'eau de rose
	\item Quelques gouttes de colorant rouge
\end{itemize}

\bigskip}
{\begin{enumerate}
	\item Épluchez les litchis, dénoyautez-les, hâchez les assez grossièrement. Rajoutez l'eau de rose dans les litchis.
	\item Faîtes fondre le beurre. Ajoutez le sucre et les \oe ufs. Bien mélanger.
	\item Versez la poudre d'amande, puis versez la farine mélangée à la levure chimique. Et encore un petit coup de fouet !
	\item Rajoutez le mélange litchis-rose. Enfin, rajoutez quelques gouttes de colorant rouge. Commencez avec 3, mélangez, puis rajouter jusqu'à obtenir une jolie couleur rose.
	\item Versez dans un moule de 24cm de long garni de papier sulfurisé (une bande dans un sens et une deuxième dans l'autre). Versez la pâte.
	\item Faites cuire à 180 degrés près d'une heure. Surveillez bien la cuisson et adapatez-la à votre propre four.
\end{enumerate}}

\bigskip
\Recet{Cake Volcanique aux Carambars}{}
{Pour un cake de 24cm}{\begin{itemize}
	\item 30 carambars + 10 en morceaux
	\item 230g de lait entier
	\item 100g de sucre roux
	\item 210g de beurre demi-sel
	\item 360g de farine
	\item 7g de levure chimique
	\item 1 œuf de 60g
\end{itemize}}
{\begin{enumerate}
	\item Placez  30 carambars dans une casserole avec le lait entier et le sucre roux. Faites fondre sur feu doux, sans jamais porter à ébullition.
	\item Ajoutez alors le beurre demi-sel et faites le fondre de la même façon, sans ébullition.
	\item Dans un récipient, mélangez la farine et la levure, puis versez l'intégralité du contenu de la casserole. Mélangez au fouet en commençant par le centre puis en allant vers la paroi du contenant. Mélangez bien pendant une minute pour n'avoir aucun grumeau. Ajoutez l'œuf et mélangez.
	\item Coupez les 10 carambars restants en morceaux et ajoutez-les à la pâte.
	\item Versez la pâte dans un moule à cake de 24cm, graissé et fariné, ou chemisé de papier sulfurisé.
	\item Faites le cuire à 160°C pendant 50 minutes. Vérifiez la cuisson et allongez le temps au besoin.
	\item Laissez-le tiédir un peu. Démoulez sur une grille à pâtisserie et dégustez-le tiède ou à température. Conservez-le impérativement emballé dans du film étirable. 
\end{enumerate}
}

\bigskip
\Recet{Carrot cake}{Préparation : 30 minutes \\ Cuisson : 40 à 50 minutes \\ Réfigiration : 5 heures}
{}{Pour le gâteau aux carottes
\begin{itemize}
	\item 1/4 de cuillerée à café de 4 épices
	\item 1/4 de cuillerée à café de gingembre en poudre
	\item 2 cuillerées à cadée de canelle en poudre
	\item 1/2 cuillerée à café de bicarbonate de soude
	\item 1 cuillerée à café de levure chimique
	\item 1/2 cuillerée à café de sel
	\item 220g de farine
	\item 150g de sucre
	\item 150g de vergeoise blonde
	\item 3 \oe ufs
	\item 1/4 de cuillerée à café de vanille liquide
	\item 220g d'huile neute
	\item 120g de noix de pécan
	\item 80g de raisins blonds
	\item 210g de carottes râpées
\end{itemize}
Pour le glaçage :
\begin{itemize}
	\item 200g de beurre doux
	\item 480g de cream cheese
	\item 280g de sucre glace
	\item 1/2 cuillère à café de vanile liquide
	\item 1 pointe d'un couteur de vanille en poudre
\end{itemize}}
{\begin{enumerate}
	\item Préchauffer le four à 180°C
	\item Dans un bol, mélanger les épices, le bicarbonate, la levure, le sel, la farine, le sucre et la vergeoise
	\item Dans un autre bol, mettez les \oe ufs, la vanille liquide et l'huile et homogénéisez-les ensemble. Versez la pâte liquide et l'huile et sur le mélange sec puis mélangez doucement avec un fouet pour obtenir une belle pâte lisse. Ajoutez ensuite les raisons et les noix de pécan concassées.
	\item Épluchez et lavez les carottes, râpez-les puis pesez-en la quantité nécessaire. Ajoutez-les à la pâte.
	\item Chemisez un grand moule à cake de papuier sulfurisé. Versez la pâte dans le moule et enfournez pour 40 à 50 minutes à 170°C. Le gâteau doit être bien cuit à l'intérieur.
	\item Sortez le gâteau et démoulez-le rapidement. Grâce au papier sulfurisé, cela ne doit poser aucun problème.
	\item Sans attendre, enveloppez-le dans du film alimentaire. De cette façon, toute la vapeur contenue dans le gâteau va rester à l'intérieur et le moelleu sera conservé. Laissez complètement refroidir.
	\item Lorsque le gâteau est à température ambiante, prépare le glaçage. Mélangez le sucre glace, le beurre doux mou et le cream cheese au batteur. Ajouter la vanille liquide et la vanille en poudre et fouettez le tout. 
	\item Coupez le gâteau en trois dans l'horizontale. Remettrez la base du gâteau dans le moule, puis étalez une couche d'une épaisseur de 1cm enciron de glaçage dessus. Placez la deuxième partie du gâteau et ajouter une deuxième couche de laçage. Ajoutez le dernier tiers du gâteau sur la couche de glaçage. Mettrez le moule au réfrigirateur deux bonnes heures pour figer la crème.
	\item Trempez la base du moule dans de l'eau chaude pour démouler facilement le gâteau. Termuinez en nappant généreusement l'extérieur du gâteau du reste du glaçage. Pour finir, vous pouvez utiliser une petite spatule pour bien lisser. Vous pouvez ensuite décorer le carrot cake avec des cerneaux de noix de pécan.
	\item Réservez au moins 3 heures avant de servir en tranches.
\end{enumerate}}

\bigskip
\Recet{Chaussons pêche-amandes}{}
{Pour 6 chaussons}{
\begin{itemize}
	\item 4 pêches jaunes mûres
	\item 30 grammes de sucre
	\item 50 grammes de poudre d’amande
	\item 30 grammes d’amandes effilées
	\item 1 gousse de vanille
	\item 1 rouleau de pâte feuilletée
	\item Lait
	\item Sucre glace
\end{itemize}}
{\begin{enumerate}
	\item Peler les pêches mûres puis les couper en morceaux. Mettre les pêches dans une casserole avec le sucre. 
	\item Couper la gousse de vanille en 2 et récupérer les graines. Ajouter les graines puis la gousse de vanille aux pêches.
	\item Faire cuire à petits bouillons pendant 5 minutes. Augmenter la puissance, faire compoter et légèrement caraméliser.
	\item Retirer la gousse de vanille et hors du feu, ajouter la poudre d’amande, remuer puis ajuster en fonction de l’humidité des fruits.
	\item Incorporer les amandes effilées, remuer et mettre au frais.
	\item Préchauffer le four à 180°c ou thermostat 6.
	\item Sortir la pâte feuilletée du frais au dernier moment. Dérouler la pâte puis découper des cercles à l'aide d'un bol par exemple. Former un dernier chausson avec les chutes de pâte.
	\item Placer 1 cuillère à soupe de compote au centre de la pâte. Refermer le chausson et collez les bords. Placez les chaussons au fur et à mesure sur une plaque avec un papier cuisson.
	\item Placez un peu de sucre glace au fond d'un verre, recouvrez de lait et remuez. Passez un coup de pinceau imbibé de ce lait sucré sur les chaussons.
	\item Placer au four pour 12-15 minutes en fonction de votre four et de la grosseur des chaussons.
\end{enumerate}}


\bigskip
\Recet{Cheesecake}{}
{Pour 6 à 8 personnes}{
\begin{itemize}
	\item 2 oeufs
	\item 300 g de golden grahams
	\item 65 g de beurre
	\item 450 g de philadelphia
	\item 150 g de sucre
\end{itemize}}
{\begin{enumerate}
	\item Préchauffer le four à 160°C.
   	\item Dans un plat à quatre quart (si vous utilisez un plat rond à charnières, doublez les quantités) réduisez les golden grahams en miettes (vous pouvez pour vous aider les mixer auparavant). Faire fondre le beurre et le rajouter aux golden grahams. Rajouter trois cuillères à soupe de sucre. Bien tapisser le fond du plat avec le mélange.
	\item Dans un bol, mélanger le sucre restant avec le philadelphia. Rajouter deux oeufs et bien mélanger.
    	\item Verser le mélange dans le plat et bien étaler.
	\item Faites cuire 45 minutes en tournant le plat à mi cuisson.
	\item Mettre au frigo pour la nuit
\end{enumerate}}

\bigskip
\Recet{Choc Chip Bars}{}
{}{
\begin{itemize}
	\item 200g de vergeoise blonde ou brune
	\item 160g de sucre
	\item 2 gros oeufs
	\item 1 cuillerée à café de vanille liquide
	\item 190g de beurre doux fondu
	\item 380g de farine
	\item 1 cuillerée à café de bicarbonate de soude
	\item 1/2 cuillerée à café de sel
	\item 300g de pépites de chocolat au lait, noir ou un mélange des deux!
\end{itemize}}
{\begin{enumerate}
	\item Mélanger dans un bol, les oeufs, le sel, le sucre blanc, la vergeoise et la vanille.
	\item Ajouter le beurre fondu et bien mélanger.
	\item Dans un autre récipient, mélanger la farine et le bicarbonate. Ajouter la farine dans le mélange aux oeufs en une fois et bien mélanger. Ajouter enfin les pépites de chocolat.
	\item Remplir un moule de $30\times30$cm de papier sulfurisé (une fois dans un sens et une fois dans l'autre) puis ajouter la pâte. 
	\item Mettre la pâte dans le moule et bien aplatir avec le dis d'une cuillère. Cuire 35 minutes à 150-160°C.
	\item Quand c'est cuit, laisser refroidir puis démouler. Couper en morceaux carrés et disperser autour d'enfants enragés.
\end{enumerate}}

\bigskip
\Recet{Crème Brûlée}{Préparation : 1h \\ Cuisson : 50 minutes \\ Repos : min. 5h}
{Pour 6 crème brûlée}{\begin{itemize}
\item 5 jaunes d'œufs
\item 50cl de crème liquide entière
\item 80g de sucre
\item 50g de lait entier en poudre
\item 1 gousse de vanille
\item 1+1/2 feuilles de gélatine
\item sucre semoule pour caraméliser
\end{itemize}}
{\begin{enumerate}
\item Préchauffer le four à 100°C.\\
Commencer par laisser tremper les feuilles de gélatine dans de l'eau froide.
\item Mélanger rapidement la moitié du sucre et les jaunes, sans les faire blanchir. Ajouter la poudre de lait entier et l'intérieur de la gousse de vanille. Bien mélanger sans s'attarder.
\item Chauffer la crème, l'autre moitié du sucre et la gousse de vanille gratée. Dès la première ébullition, enlever du feu et verser sur le mélange jaunes/sucre/poudre de lait.
\item Remettre sur le feu doux et cuire comme une crème anglaise, c'est à dire ne pas la faire bouillir et arrêter quand celle-ci nappe la cuiller en bois d'une couche de crème. Enlever du feu, enlever la gousse de vanille, passer un coup de mixer plongeant dans la crème pendant trois ou quatre minutes (cela va uniformiser la crème) puis ajouter la gélatine bien essorée et bien mélanger.
\item Verser la crème dans des ramequins. Cuire au four à 100°C pendant 50 minutes.
\item Une fois les crèmes refroidies, les placer au réfrigérateur pendant au moins 5 heures, le mieux étant une nuit entière !
\item Avant de servir, saupoudrer de sucre toute la surface de la crème, puis brûler avec le chalumeau toute la surface. On peut remettre une petite demi-heure au frais pour raffermir de nouveau la crème mais on peut tout aussi bien la servir tout de suite !
\end{enumerate}}

\bigskip
\Recet{Crêpes mille-trous}{}
{}{\begin{itemize}
	\item 370g d'eau froide
	\item 215g de semoule fine
	\item 25g de farine
	\item 10g de sucre
	\item deux bonnes pincées de sel
	\item 1 sachet de levure de boulanger déshydratée
	\item 2 cuillerées à café de levure chimique
\end{itemize}}
{\begin{enumerate}
	\item Mettre l'eau dans le récipient du blender avec la levure de boulanger. Ajouter la semoule, ainsi que la farine, le sel et le sucre. Bien mixer pendant au moins 2-3 minutes pour obtenir une pâte lisse. De plus, en mixant, on va chauffer l'ensemble, d'où l'importance de mettre de l'eau froide. À la fin des 3 minutes, la pâte est tiède voire commence à chauffer.
	\item Laisser lever 15 minutes, la pâte va gonfler et former plein de petites bulles. Si la pâte ne bulle pas trop, ce n'est pas très grave! Ajouter à ce moment les deux cuillerées à café de levure chimique puis remettre à mixer 20 secondes.
	\item Chauffer une petite poêle sur feu assez fort. Verser une petite louchée de pâte. La crêpe va commencer à buller, mais il ne faut faire cuire que d'un seul côté: de cette façon, les crêpes restent bien moelleuses. Quand toute la pâte apparait cuite (ce qui n'est pas le cas encore complètement sur la photo!), il suffit de la sortir puis de recommencer avec une nouvelle louchée de pâte!
	\item Servir les baghrirs avec du miel chaud et du beurre... 
\end{enumerate}}

\medskip
\Recet{Far breton}{Préparation : 20 minutes \\ Cuisson : 60 minutes}
{Pour 6 à 8 personnes}{\begin{itemize}
\item 220 g de farine
\item 130 g de sucre en poudre
\item 1 sachet de sucre vanillé
\item 50cl de lait
\item 25cl de crème fraiche entière
\item 5 oeufs
\item 20 g de beurre
\item 500 g de pruneaux (Facultatif)
\end{itemize}}
{\begin{enumerate}
\item Pour commencer, préchauffez le four à 180°C (thermostat 6).
\item Dans un saladier, mélangez le sucre, la farine et ajoutez le sucre vanillé. Puis ajoutez les œufs en prenant soin de bien mélanger délicatement le tout à chaque fois. Versez le lait et ajoutez le beurre au préalablement fondu puis mélangez jusqu'à ce que vous obteniez une pâte homogène.
\item Ajoutez vos pruneaux si vous souhaitez obtenir un far aux pruneaux et pensez à les dénoyauter s'ils ont des noyaux (mais vous pouvez évidemment le déguster nature -ce qui est d'ailleurs plus traditionnel- ou avec des pommes \bcinfo).
\item Beurrez le fond de votre moule et versez y la pâte.
\item Vous pouvez placer votre moule au four et patientez une heure environ.
\end{enumerate}
\bcinfo\ Pour un far aux pommes, faîtes revenir à la poêle 500g de pommes coupées en morceaux dans 30g de beurre.}

\medskip
\Recet{Flan Parisien}{Préparation : 1h30 \\ Cuisson : 45 minutes \\ Réfrigération : 3h30}
{Pour un flan de 24cm}{Pour la pâte fécule \begin{itemize}
\item 250g de farine de type 55
\item 50g de fécule de pomme de terre
\item 225g de beurre
\item 55g de lait
\item 15g de jaune d'\oe uf
\item 5g de sel
\item 30g de sucre
\end{itemize}
Pour la crème à flan
\begin{itemize}
\item 180g de sucre
\item 130g de fécule de maïs
\item 2 goutte de vanille liquide
\item 160g d'\oe ufs
\item 60g de jaunes d'\oe ufs
\item 30g de beurre doux
\item 1,3l de lait demi-écrémé\\+135g de sucre semoule
\end{itemize}}
{\begin{enumerate}
\item Préparez la pâte décule : mettez le beurre froid coupé en morceaux, la farine, la fécule, le sel et le sucre dans un bol de robot mélangeur muni de la feuille. On peut bien sûr pétrir à la main. Mélangez jusqu'à obtenir une poudre sableuse.
\item Dans un petit bol, fouettez ensemble le jaune d'\oe uf et le lait. Versez sur le mélange sableux et remuez jusqu'à ce que la pâte soit homogène. La pâte va sans doute paraître un peu molle, surtout si vous faîtes cette recette un jour où il fait chaud. Mettez la pâte sous film étirable et stockez-la 30 minutes au frais.
\item Étalez la pâte sur un plan de travail légèrement fariné ou sur un papier sulfurisé sur une épaisseur de 2mm. S'il fait chaud, procédez par étapes ! Commencez à étaler et si vous voyez que vous avez du mal mettez la pâte au frais quelques minutes.
\item Foncez un cercle à entremet de 24cm de diamètre posé sur une plaque à pâtisserie garnie de papier sulfurisé (ou un moule à bords haut, mais le resultat sera un peu moins beau !). Assurrez-vous de bien plaquer la pâte sur le fond et sur le bord du cercle. Piquez le fond de pâte puis mettez au réfrégirétaeur jusqu'à utilisation.
\item Préchauffez le fout à 180°C
\item Préparez maintenant la crème à flan : mettez les 180g de sucre, la vanille et la fécule de maïs dans un récipient. Ajoutez les \oe ufs entiers et les jaunes. 
\item Portez le lait avec le beurre et les 135g de sucre à ébullition dans une grande casserole sur feu doux. Quand le lait bout, versez en filet sur le mélange précédeux tout en fouettant. Mélangez bien puis reversez le tout dans la casserole.
\\ Laissez bouillir sur feux doux 3 à 4 minutes toujours en mélangeant avec un fouet.Versez l'intégralité de la crème dans le cercle foncé. Lissez la surface de la crème avec une spatule puis en secouant légèrement la plaque.
\item Enfournez à 180°C pendant 40 à 50 minutes. Le flan doit ressortir bien gonflé et coloré. Laissez-le complètement refroidir, puis décerclez le soigneusement. Mettez le alors au frais pendant au moins 3 heures.
\end{enumerate}}

\bigskip
\Recet{Fondant au Chocolat et Crème de Marrons}{}
{}{\begin{itemize}
	\item 150g de chocolat noir
	\item 50g de beurre demi-sel
	\item 80g de beurre doux
	\item 500g de crème de marrons (ce n'est pas la purée)
	\item 4 œufs
	\item 40g de sucre glace
\end{itemize}}
{\begin{enumerate}
	\item Préchauffez votre four à 150°C.
	\item Mettez les beurres avec le chocolat à fondre au four à micro-ondes ou au bain-marie.
	\item Mettez la crème de marrons dans un grand bol. Ajoutez les œufs et mélangez bien. Versez le chocolat, le beurre fondus, puis le sucre glace. Mélangez pour obtenir une pâte bien lisse. 
	\item Chemisez un moule à manqué à bord amovible de 20cm de diamètre. Vous pouvez également prendre un moule normal de 20cm mais le démoulage sera un peu plus délicat. Si vous souhaitez un fondant plus fin, prenez alors un moule de 22cm.
	\item Versez la pâte. Enfournez et laissez cuire 45 minutes. Le gâteau va gonfler puis retomber après cuisson. Laissez-le refroidir totalement avant de le mettre au frais. 
	\item Démoulez et servez en fines tranches ! 
\end{enumerate}}

\bigskip
\Recet{Gâteau roulé}{\label{gateauRoule}Préparation : 20 minutes \\ Cuisson : 10 minutes}
{Pour 6 personnes}{\begin{itemize}
	\item 4 \oe ufs
	\item 125g de farine
	\item 100g de sucre
	\item 1 cuillerée à café de levure chimique (à voir : remplacer par du bicarbonate)
	\item 1 cuillère à café de fleur d'oranger.
	\item Garniture au choix (Gelée, confiture, crème chantilly avec fruits...)
\end{itemize}}
{\begin{enumerate}
	\item Allumer le four thermostat 7 / 210°C
	\item Séparez les blancs des jaunes. Travaillez les jaunes avec le sucre jusqu'à ce que le mélange devienne blanc. Ajoutez la farine petit à petit puis la levure.
	\item Battez les blancs en neige et incorporez-les au mélange. Ajouter la fleur d'oranger. Ne pas mélanger trop longtemps ou la pâte se liquéfiera.
	\item Beurrez un moule rectangulaire et y verser la pâte. Faire cuire à four chaud pendant 8 minutes.
	\item Démoulez le gâteau, laissez-le se refroidir un peu (surtout si vous y rajouter une crème du type crème au beurre ! Avec la chaleur, la crême coulera et le résultat sera d'autant moins appréciable. Dans ce cas, attendez une heure avec un torchon humide au-dessus pour qu'il ne durcisse pas). 
	\item Étalez la garniture et roulez le gâteau avec précaution. 
	\item Faîtes une décoration (le sucre glace, c'est pas mal, mais à faire à froid, ou le sucre se dissolvera !), et dégustez !
\end{enumerate}}

\bigskip
\Recet{Havreflarns}{}
{Pour une cinquantaine de galettes}{Pour les galettes :
\begin{itemize}
\item 200g de flocons d'avoine
\item 300g de sucre
\item 180g de matière grasse de cuisine
\item 70g de farine
\item 20g de noix de coco râpée (facultatif)
\item 1 ou 2 gouttes de vanille liquide
\item 60g d'\oe uf (1 très gros \oe uf)
\item 1 bonne pincée de sel
\item 1/4 de cuillerée à café de levure chimique
\item 1/4 de cuillerée à café de de bicarbonate de sodium
\end{itemize}
Pour les galettes au chocolat :
\begin{itemize}
\item 300g de chocolat noir ou au lait
\end{itemize}}
{\begin{enumerate}
\item Dans un récipient, placez les flocons d'avoine, le sucre, la farine, la noix de coco et le sel. Mélanger sommairement.
\item Faître fondre la matière grasse de cuisoon (à la casserole ou au micro-ondes). Attention aux éclaboussures ! \\
Versez-la dans le récipient puis ajoutez l'\oe uf et la vanille liquide.
\item Laissez refroidir quelques minutes (si la matière grasse était trop chaude) puis ajouter la levure chimique et le bicarbonate.
\item Mélangez parfaitement la pâte puis mettez-la au réfrigérateur pendant au moins 2h.
\item Après ce temps, déalisez des boulettes de pâte. Il veut mieux les peser pour avoir un résultat parfait : 8g est l'idéal !
\item Préchauffer le four à 180°C. Placez les boulettes sur une plaque à pâtisserie garnie de papier suflurisé en les espaçant suffisamment car elles vont s'étaler. Faîtes cuire chaque fournée 8 à 9 minutes. Les galettes doivent être bien dorées sur le pourtour et le centre un peu plus clair. Transvasez sur une grille avec une spatule le temps qu'elles refroidissent et qu'elles durcissent.
\item Placez les havreflarns dans une boîte hermétique pour bien les conserver : l'humidité est la pire ennemie de ces petits gâteaux !

\bigskip
Pour les doubles galettes au chocolat :
\item Faire fondre le chocolat au bain-marie ou au micro-ondes (sans ajouter de chocolat dedans !). Si vous faîtes cela au micro-ondes, mettez le chocolat à fondre 20 secondes par 20 secondes en mélangeant à chaque fois.\\
Prenez une galette et posez-la à plat dans le chocolat fondu. Enfoncez un peu avec une fourcette jusqu'à ce que le bord touche le chocolat. Puis reprenez la galette et retournez la sur une grille. Prenez ensuite une autre galette et posez-la sur celle couverte de chocolat. Laissez bien refroidir et replacez les galettes dans une boîte hermétique. Ils se conservent un mois sans problème s'ils sont à l'abri de l'air et de l'humidité !
\end{enumerate}}

\bigskip
\Recet{Mi-cuits au chocolat}{}
{Pour 4 mi-cuits}{\begin{itemize}
	\item 100g de chocolat à 65%
	\item 100g de sucre glace
	\item 100g de beurre salé
	\item 2 oeufs + 2 jaunes
	\item 50g de farine
\end{itemize}}
{\begin{enumerate}
	\item Mélanger les jaunes, les oeufs et le sucre glace sans blanchir le tout. On veut juste les mélanger sinon la texture des gâteaux sera trop légère et ces gâteaux sont bons car ils n'ont rien de léger !
	\item Faire fondre le chocolat et le beurre au micro-ondes ou au bain-marie. Ajouter le chocolat/beurre fondu au mélanger sucre/oeufs. Mélanger sans insister.
	\item Ajouter ensuite la farine au mélange. Bien mélanger.
	\item Graisser 4 petits ramequins en alluminium. Verser la pâte. Laisser reposer au frais le temps de préchauffer le four à 200°C.
	\item Cuire les gâteaux jusqu'à ce qu'une croûte se soit formée sur le dessus si l'on aime les gâteaux très coulants (envrion 8 minutes), soit un peu plus pour rafermir l'intérieur du gâteau. Tout l'întéret de ces gâteaux, réside dans cet intérieur très coulant. Sortir les gâteaux du four et laisser reposer 1 minute avant de les renverser chacun sur une assiette.
	\item Démouler et servir aussitôt! La cuiller va casser la légère croûte et l'intérieur va commencer à couler! 
\end{enumerate}}

\medskip
\Recet{Millionaire shortbreads}{Préparation : 40 minutes \\ Cuisson : 40 minutes \\ Réfrigération : 2h}
{Pour un moule de 21 $\times$ 21cm}{Pour la pâte à shortbreads\begin{itemize}
\item 60g de sucre
\item 120g de beurre saké
\item 180g de farine
\end{itemize}
Pour le caramel \begin{itemize}
\item 150g de vergeoise blonde
\item 25g de miel
\item 175g de beurre
\item 1 boîte de lait concentré
\item 1/4 de cuillère à café de sel
\item 200g de chocolat au lait
\end{itemize}}
{\begin{enumerate}
\item Préchauffez votre four à 180°C
\item Mettez le beurre salé mou et le sucre dans un bol. Mélangez pour obtenir une sorte de crème (avec la feuille si vous avez un robot !). Ajoutez la farine. Mélangez et ramenez la pâte en boule sans travailler de façon excessive. Mettez la boule dans un moule de 21 $\times$ 21cm garni de papier sulfurisé. Étalez uniformément à la main puis avec le dos d'une cuillère à soupe pour avoir une pâte bien régulière. Mettez au four pendant 25 minutes.
\item Préparez le caramel en mettant la vergeoise, le sel, le miel et le beurre dans une casserole. Mettez sur feu doux et portez à ébullition. Laissez bouillir 3-4 minutes. Ajoutez ensuite la boîte de lait concentré sucré. Mélangez bien puis laissez cuire 8-10 minutes à partir de l'ébullition sur feu doux tout en mélangeant constamment.
\item Versez le caramel sur la pâte cuite, toujours dans le moule carré. Laissez refroidir totalement puis mettez 2 heures au frais.
\item Mettez le chocolat à fondre puis versez-le sur le caramel. Répartissez-le bien en remuant le moule pour avoir une surfasse lisse. Quand le chocolat commence à prendre mais qu'il n'est pas encore dur (10 minutes après le versement à peu près), coupez des parts avec une lame bien coupante trempée dans de l'eau chaude.
\item Conservez en boîte hermétique.
\end{enumerate}}

\bigskip
\Recet{Pumpkin bread (Pain à la cirtouille)}{Préparation : 30 minutes \\ Cuisson : 1 heure 30}
{Pour un bon pain}{\begin{itemize}
	\item 220 g de farine 
	\item 80 ml d'eau 
	\item 2 oeufs
	\item 300 g de sucre 
	\item 1/2 cuillère à café de levure
	\item 1/2 cuillère à café de levure alsacienne
	\item 1/2 cuillère à café de noix de muscade
	\item 1/2 cuillère à café de cannelle
	\item 120 ml d'huile neutre
	\item 245 g de purée de potiron
	\item 1 cuillère à café de sel
\end{itemize}}
{\phantom{.}

\bigskip
\begin{enumerate}
	\item Préchauffer le four à 175 degrés C 
	\item Dans un bol, mélanger la farine, les levures, le sel, la cannelle, le 4 épices, la noix de muscade et les clous de girofle
	\item Dans un autre bol, battre au batteur le sucre, les oeufs et l'huile
	\item Verser petit à petit la purée de citrouille tout en battant
	\item Rajouter en battant le mélange de farine
	\item Verser la pâte dans le moule
	\item Faire cuire 90 minutes
	\item Laisser refroidir 10 minutes avant de démouler
\end{enumerate}

\bigskip
\phantom{.}}

\bigskip
\label{TarteRhubarbe}
\Recet{Tarte à la rhubarbe et à la poudre d'amandes}{Préparation : 20 minutes \\ Cuisson : 40 minutes}
{Pour 6 personnes}{\begin{itemize}
	\item 500 g de Rhubarbe 
	\item 400 g de pâte sablée
	\item 100 g de beurre
	\item 100 g de sucre en poudre
	\item 100 g de cassonade
	\item 125 g de poudre d'amandes
	\item 2 oeufs
	\item 2 c. à soupe de farine
	\item 1 c. à café rase de cannelle en poudre
	\item 1 pincée de gingembre moulu
\end{itemize}}
{\begin{enumerate}
	\item Epluchez et coupez la rhubarbe en tronçons.
	\item Plongez-les une minute dans l’eau bouillante, puis égouttez-les.
	\item Mélangez 75 g de cassonade et les épices.
	\item Mettez la rhubarbe dans une casserole et saupoudrez-la du mélange sucre/épices.
	\item Placez la casserole sur feu doux et faites cuire 5 min en remuant.
	\item Retirez du feu, et laissez tiédir.
	\item Préchauffez le four th.6 (180°C).
	\item Mettez le beurre, le sucre, la poudre d’amandes, la farine et les œufs entiers dans le bol du mixeur.
	\item Faites tournez très rapidement jusqu’à obtention d’une crème.
	\item Etalez la pâte sur un plan de travail fariné et garnissez-en un moule à tarte beurré. Piquez le fond avec une fourchette et placez au frais.
	\item Mélangez la rhubarbe et la crème d’amande.
	\item Versez la préparation dans le fond de tarte.
	\item Enfournez et faites cuire 35 min.
	\item 10 min avant la fin de la cuisson, saupoudrez la tarte du reste de cassonade.
	\item Sortez la tarte du four et laissez refroidir avant de servir.
\end{enumerate}}

\medskip
\phantom{.}

\bigskip
\Recet{Tarte aux fraises sur lit de pistache}{Préparation : 1h30 \\ Cuisson : 20 minutes \\ Réfrigiration : 3 à 12h \\ \scriptsize{(La crème pâtissière peut se faire la veille)}}
{Pour 6 personnes}{\begin{itemize}
\item 1 pâte sablée (voir la recette de la \hyperref[TarteRhubarbe]{tarte à la rhubarbe})
\item 250g de fraises
\end{itemize}
Pour la crème de pistache :
\begin{itemize}
\item 50g de beurre fondu
\item 50g de poudre de pistaches (ou de pistaches fraîches réduits en poudre)
\item 50g de sucre glace
\item 5g de fécule de maïs (Maïzena)
\item 30g d'\oe uf
\item 10g de kirsch
\end{itemize}
Pour la crème patissière :
\begin{itemize}
\item 250ml de lait demi-écrémé
\item 25g de beurre doux
\item 25g de fécule de maïs
\item 50g de jaunes d'\oe ufs
\item 60g de sucre
\item 1 pincée de vanille en poudre
\end{itemize}
}{\begin{enumerate}
\item Commencez par préparer la crème pâtissière. Le mieux étant de la faire la veille pour qu'elle soit bien froide au moment de son utilisation. Mettez les 50g de jaunes d'\oe ufs dans un récipient avec la vanille en poudre, la moitié du sucre et la fécule. Mélangez au fouet. Faîtes bouillir le lait avec l'autre moitié du sucre et le beurre.\\
Quand le lait bout, versez-le en filet sur le mélange jaunes-poudre à crème tout en fouettant. Mélangez bien puis reversez cette préparation dans la casserole. Remettez à cuire sur feu doux.\\
Laissez bouillir 3 à 4 minutes toujours en mélangeant vivement.\\
Versez la crème pâtissière dans un plat puis couvrez d'un film étirable directement au contact. Laissez refroidir complètement puis placez au réfrigérateur pour 3 heures ou pour une nuit entière.
\item Le jour même, étalez votre pâte sur 2 ou 3mm d'épaisseur puis foncez un cercle de 18cm. Mettez le tout au réfrigérateur. Préchaufez votre four à 180°C.
\item Préparez la crème de pistaches. \\
Dans un bol, mettez la poudre de pistache, le sucre glace, la fécule et le beurre mou. Ajoutez le kirsch et les 30g d'\oe uf, puis mélangez au fouet. Versez la crème d'amande dans le cercle.
\item Enfournez la tarte pour une vingtaine de minutes. Il faut adapter le temps de cuisson à ce que vous voyez ! La tarte doit être bien dorée dessus et dessous. Laissez refroidir sur une grille à pâtisserie.
\item Dans un grand bol, mettz la crème pâtissière bien froide puis fouettez la pendant 1 ou 2 minutes pour bien la lisser et l'assouplir.\\
Versez-la dans la tarte sur la crème de pistaches.
\item Coupez les fraises en deux en enlevant la queue puis placez-les sur la tarte. Mettez au frais jusqu'au moment de servir !
\end{enumerate}}

\bigskip
\Recet{Tarte aux noix au caramel}{Préparation : 30 minutes \\ Cuisson : 30 minutes}
{Pour 8 personnes}{Pour la pâte sâblée :
\begin{itemize}
	\item 220 g de farine 
	\item 20g de poudre de noisettes
	\item 125 g de beurre 
	\item 70 g de sucre semoule ou glace 
	\item 1 oeuf 
	\item 5 cl d'eau (ou de lait) 
	\item 1 pincée de sel 
\end{itemize}
Pour la tarte :
\begin{itemize}
	\item 250 g de cerneaux de noix
	\item 80 g de sucre
	\item 1 oeuf
\end{itemize}
Pour le caramel :
\begin{itemize}
	\item 100 g de sucre
	\item 100 g de crème fraîche(10 cl)
	\item 1/2 verre d'eau
\end{itemize}}
{\begin{enumerate}
	\item Préparer la pâte sablée : fouettez l'\oe uf avec le sucre et détendre avec un peu de lait
	\item Mettre le beurre en parcelles et le sel sur la farine et la poudre de noisettes.
	\item Écraser beurre et farine ensemble en frottant légèrement les mains. La farine passe entre les doigts.
	\item Le sablage terminé formez une fontaine. Verser au centre le mélange sucre + oeufs + eau.
	\item Serrer doucement la masse entre les mains. Farinez légèrement le plan de travail et fraiser jusqu'a obtention de l'amalgame. Former une boule.
	\item Préchauffer le four à 180°C. Étalez la pâte sur du papier sulfurisé, et la mettre dans un moule. Recouvrir le fond de pâte d'un rond de papier sulfurisé et remplir de haricots secs ou de noyaux de fruits. 
	\item Glisser la pâte au four et faire cuire 20 minutes. Mixer 50 g de noix avec 80 g de sucre et l'oeuf battu. Verser sur la pâte débarassée des haricots et du rond de papier sulfurisé. Mettre au four à 150°C (thermostat 5) une dizaine de minutes environ.
	\item Préparer le caramel mou : chauffer 100 g de sucre avec l'eau. Quand il caramélise, ajouter la crème et fouetter très fort quelques minutes.
	\item Garnir la tarte avec les cerneaux de noix. Les napper de caramel mou tiède. Laisser refroidir avant de servir.
\end{enumerate}}

\medskip
\phantom{.}

\medskip
\Recet{Tiramisu}{Préparation : 30 minutes \\ Réfrigération : 3h}
{Pour 6 personnes}{\begin{itemize}
\item 500g de mascarpone
\item 2 gros \oe ufs
\item 80g de sucre semoule
\item 30g de sucre glace
\item 2 paquets de biscuits à la cuiller
\item 1 tasse de café bien fort
\item 2 à 3 cuillères à soupe d'amaretto ou de kirsch (Facultatif)
\item Cacao en poudre (non sucré de préférence)
\end{itemize}}
{\begin{enumerate}
\item Commencez par séparer les jaunes des blancs. Mettez les jaunes avec  le sucre semoule et fouettez le tout avec un batteur électrique pour blanchir les jaunes le plus possible. Ajoutez le mascarpone et continuez de fouetter pour obtenir une crème aérée, une quasi-chantilly.
\item Dans un autre récipient, montez les blancs en neige bien ferme en ajoutant, à mi-parcours, le sucre glace. Ajoutez ces blancs à la crème au mascarpone. Mélangez délicatement avec une maruse en soulvant la crème pour ne pas faire retomber les blancs.
\item Préparez un café assez fort et ajoutez-y l'amaretto. Trempez un à un les bisctuis dans le café pendant 4 à 5 secondes pour bien les imbiber, mais sans les détremper ! Placez les bisctuis dans un plat de service de 18 sur 24cm en les serrant bien. Versez dessus la crème au mascarpone.
\item Laissez reposer au moins 3 heures au réfriférateur. Au dernier moment, saupoudrez de cacao avec un petit tamis. 
\end{enumerate}}

\bigskip
\Recet{Trianon}{Préparation : 30 minutes \\ Repos : minimum 3h}
{Pour 12 personnes}{Pour la dacquoise à la noisette :
\begin{itemize}
	\item 90 g de poudre de noisettes
	\item 90 g de sucre glace
	\item 40 g de sucre en poudre
	\item 4 blancs d'oeufs battus en neige
\end{itemize}
Pour le feuilleté praliné :
\begin{itemize}
	\item 100 g de chocolat noir
	\item 100 g de chocolat au pralin
	\item 40 g de pralin
	\item 18 crêpes dentelles brisées
	\item 10 cl de crème liquide
\end{itemize}
Pour la mousse au chocolat :
\begin{itemize}
	\item 360 g de chocolat noir (de couverture si possible)
	\item 50 cl de crème liquide entière montée en chantilly
	\item 10 cl de lait
\end{itemize}}
{\phantom{.}

\medskip
\begin{enumerate}
	\item Pensez de suite à mettre un saladier et les fouets d'un batteur au congélateur pour la crème fouettée
	\item Dacquoise à la noisette : dans un saladier, mélangez la poudre de noisette, le sucre glace e le sucre en poudre. Dans un second saladier, battez les blancs en neige puis incorporez-y les poudres peu à peu tout en soulevant la masse délicatement.\\
Versez cette pâte en fine couche dans un cercle à pâtisserie de 26 à 28 cm de diamètre (ou un rectangle de 20$\times$23 cm environ ou dans un moule à fond amovible) et faites cuire 10 à 15 minutes à 170°c ou 180°c (thermostat 6). Surveillez le temps de cuisson. 
	\item Feuilleté praliné : faites fondre les 200g de chocolats, laissez refroidir un peu et ajoutez-y la crème fraîche, le pralin et les crêpes dentelles brisées. Mélangez et étalez cette pâte sur la dacquoise refroidie. 
	\item Mousse au chocolat : faites fondre le chocolat. \\
Montez la crème liquide en chantilly. Mélangez le chocolat avec le lait, puis ajoutez le mélange à la crème fouetée. Fouettez bien le tout et versez la mousse sur le feuilleté praliné.
	\item Laisser cet entremet au frais pour la nuit ou pour la journée puis décorez-le avec du cacao amer en poudre. Pensez à le sortir 1/2 heure à 1 heure avant la dégustation.
\end{enumerate}

\medskip
\phantom{.}}


