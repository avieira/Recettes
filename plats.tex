\Recet{Penne aux courgettes}{Préparation : 20 minutes\\Cuisson : 10 minutes}
{Pour 1 personne}{\begin{itemize}
	\item 50 g de pâtes (penne ou autre)
	\item 1 courgette
	\item 1 petite boîte de tomates pelées (ou en morceaux)
	\item origan
	\item thym
	\item 1 feuille de laurier
	\item 2 cuillères à café d'huile
	\item 1 cuillère à soupe de parmesan
	\item sel, poivre
\end{itemize}}
{\phantom{.}

\bigskip
\begin{enumerate}
	\item Mettez à cuire les pâtes.
	\item Lavez et coupez la courgette en dés. Faire suer la courgette dans l'huile un instant, puis ajoutez 1 pincée d'origan, de thym et la feuille de laurier. Versez les tomates coupées en dés. Laissez cuire jusqu'à ce que les courgettes soient cuites. Assaisonner un peu selon votre goût.
	\item Egouttez les pâtes et disposez les dans une jolie assiette. Versez les courgettes dessus. Saupoudrez de parmesan.
\end{enumerate}

\bigskip
\phantom{.}}

\bigskip
\Recet{Poulet à la vietnamienne}{Préparation : 20 minutes \\ Cuisson : 15 minutes}
{Pour 4 personnes}{\begin{itemize}
	\item 800 g d'escalopes de poulet
	\item 2 oignons
	\item 1 verre de vin blanc
	\item sauce au soja
	\item gingembre moulu
	\item sel et poivre
\end{itemize}}
{\phantom{.}

\bigskip
\begin{enumerate}
	\item Découper en lanières le poulet, les saupoudrer de gingembre.
	\item Faire revenir les oignons émincés dans une sauteuse. Lorsque ceux-ci sont transparents, ajouter le poulet 
	\item Faire revenir 2 mn puis ajouter 2 cuillères de sauce au soja, le vin, sel et poivre et laisser cuire à couvert 10 mn.
\end{enumerate}

\bigskip
\phantom{.}}

\bigskip
\Recet{Filet mignon poêlé au vinaigre balsamique}{Préparation : 5 minutes \\ Cuisson : 30 minutes}
{Pour 5 personnes}{\begin{itemize}
	\item 1 filet mignon
	\item vinaigre balsamique
	\item miel
	\item crème fraiche liquide
	\item sel/poivre
	\item 1 noisette de beurre
\end{itemize}}
{\phantom{.}

\bigskip
\begin{enumerate}
	\item Faites revenir le filet mignon dans une poêle, avec 1 noisette de beurre.
	\item Ajouter le vinaigre balsamique, environ 20 ml, et le miel (3 bonnes cuillères à soupe). Saler et poivrer.
	\item Dès que la cuisson est bonne, retirer le filet mignon de la poêle, allonger la sauce avec de la crème fraiche liquide, et la servir pour accompagner votre filet mignon.
\end{enumerate}

\bigskip
\phantom{.}}

\bigskip
\Recet{Aubergines au chèvre chaud}{Préparation : 20 minutes\\Cuisson : 30 minutes}
{Pour 4 personnes}{\begin{itemize}
	\item 2 aubergines
	\item 3 petits fromages de chèvre mi-secs
	\item 6 cuil. à soupe d'huile d'olive
	\item 1 pincée de piment d'Espelette
	\item sel, poivre
\end{itemize}}
{\phantom{.}

\bigskip
\begin{enumerate}
	\item Epluchez les aubergines en laissant une bande de peau sur deux. Coupez-les en 18 tranches. Faites-les cuire 6 min à la poêle dans l'huile d'olive. Salez, poivrez.
	\item Superposez les tranches 3 par 3 en y intercalant les fromages de chèvre coupés en fines lamelles. Déposez-les dans un plat à rôtir. Enfournez 5 min th. 6/7 (200 °C). Relevez de piment d'Espelette. Servez sans attendre.
\end{enumerate}

\bigskip
\phantom{.}}

\bigskip
\Recet{Ratatouille niçoise}{Préparation : 30 minutes \\ Cuisson : 1h15}
{Pour 6 personnes}{\begin{itemize}
	\item 1kg de tomates
	\item 3 poivrons
	\item 500g de courgettes
	\item 2 aubergines
	\item 3 oignons
	\item 3 gousses d'ail
	\item 1 feuille de laurier
	\item 1/2 verre d'huile d'olive
	\item 1 pointe de cayenne
	\item 1 clou de girofle
	\item sel, poivre
\end{itemize}}
{\phantom{.}

\bigskip
\begin{enumerate}
	\item Coupez les poivrons en lamelles
	\item Épluchez et coupez les aubergines en dés
	\item De même avec les courgettes
	\item Émincez les oignons
	\item Dans un grand fait-tout, faîtes chauffer l'huile et mettez-y les oignons, les dés de courgettes et d'aubergines. Lorsqu'ils sont dorés, retirez-les et reservez-les.
	\item Dans le même fait-tout, faîtes également revenir l'ail et les poivrons. Au bout de 5 minutes, mettez les tomates. Laissez revenir une dizaine de minutes en remuant, puis ajoutez les courgettes et les aubergines reservées. Assaisonnez (pensez au clou de girofle !) et laissez cuire à couvert et à feu doux pendant 1 heure.
\end{enumerate}}

\bigskip
\Recet{Chili con carne}{Préparation : 1h30 \\ Cuisson : 2h}
{Pour 6 personnes}{\begin{itemize}
	\item 1 kg de paleron (viande qui devient plus moelleuse après une longue cuisson)
	\item 1 kg d'oignon frais
	\item 1 boîte de 500 g de haricots rouges sec
	\item 3 piments oiseaux
	\item 1 grosse boîte de maïs
	\item 2 boîtes de tomates pelées
	\item Clous de Girofle
	\item Huile d'olive
	\item vin blanc sec
\end{itemize}}
{\begin{enumerate}
	\item Faire tremper les haricots pendant au moins une nuit. Une fois le trempage effectué, rincer 3 fois minimum les haricots.
	\item Faire cuire les haricots (avec un gros oignon piqué de 2 clous de girofle) jusqu'à ce qu'ils aient la consistance des châtaignes cuites. Ils ne doivent surtout pas éclater car ils cuiront 2 bonnes heures par la suite. Une fois cuits, mettez-les de côté.
	\item Faire revenir les piments dans de l'huile d'olive. Il faut que les piments soient presque cramés. Je dis bien presque ! Jetez les piments : seule l'huile nous intéresse.
	\item Faites revenir la viande à feu très vif dans cette huile.
	\item Faites ensuite revenir les oignons dans cette même huile (ce qu'il reste, mais vous pouvez en ajouter un peu si vous voulez). Ils faut qu'ils soient bien dorés. (Vous pouvez ajouter un peu de vin blanc sec de temps en temps si vous le souhaitez)
	\item Quand vos oignons ont fini de cuire dans la sauce de la viande (qui a elle même cuit dans l'huile des piments), vous jetez dedans le contenu de 2 boîtes de tomates pelées. Si vous aimez vraiment la tomate, vous pouvez sans crainte ajouter 1 boîte de concentré. Laissez réduire.
	\item Quand vous trouvez que tout ça c'est pas mauvais du tout, ajoutez la viande et les haricots.
	\item Vous laissez mijoter assez longtemps (2h minimum, en principe). Quand vous sentez que c'est presque cuit, ajoutez une grosse boîte de maïs. Laissez cuire encore 10 min.
\end{enumerate}}

\bigskip
\Recet{Risotto au poulet}{Préparation : 35 minutes\\Cuisson : 1 heure}
{Pour 6 personnes}{\begin{itemize}
	\item 500 g de riz Arborio
	\item 3 blancs de poulet
	\item 200 g de lardons fumés
	\item 200 g de petits pois
	\item 15 cl de vin blanc sec
	\item 1,5 l de bouillon de volaille
	\item 10 cl d'huile d'olive
	\item 20 g de beurre
	\item 20 cl de crème fraîche (facultatif)
	\item 4 échalotes
	\item 2 gousses d'ail
\end{itemize}}
{\begin{enumerate}
	\item Chauffer le bouillon de volaille (ou 1,5 l d'eau et 3 cubes). Cuire les petits pois 10 minutes à l'eau bouillante légèrement salée et additionnée d'une pincée de sucre. Ebouillanter les lardons 1 minute pour les dessaler.
	\item Pendant ce temps, éplucher et émincer l'ail et les échalotes. Découper le poulet en fins bâtonnets ou en lamelles.
	\item Chauffer l'huile et le beurre dans une sauteuse et y faire revenir les lamelles de poulet et les lardons 2 minutes. Les retirer et réserver au chaud. Ajouter dans la sauteuse l'ail et les échalotes. Faire revenir 1 minute sans les brûler.
	\item Ajouter le riz en remuant à la cuillère en bois pour bien imprégner chaque grain de matière grasse. Déglacer avec le vin blanc. Porter à ébullition. Ajouter une louche de bouillon de volaille bien chaud tout en remuant. Lorsque cette louche de bouillon est absorbée par le riz, en ajouter une autre tout en remuant et ainsi de suite jusqu'à utilisation de tout le bouillon. Avec la dernière louche de bouillon, ajouter le poulet, les lardons et les petits pois égouttés.
	\item Juste avant de servir, ajouter la crème fraîche tout en continuant à remuer. 
\end{enumerate}}

\bigskip
\Recet{Lasagnes}{Préparation : 1h30 \\ Cuisson : 35 minutes}
{Pour 6 personnes}{\begin{itemize}
	\item 1/2 paquet de pâte à lasagnes
	\item 250 g de viande hachée de boeuf
	\item 150 g de chair a saucisse
	\item 1 boîte de tomates en dés
	\item 1 bouteille de purée de tomate (passata)
	\item 1 oignon
	\item 1 carotte
	\item 1 branche de céleri
	\item huile d'olive
	\item sel, poivre
	\item basilic (frais si possible)
	\item mozzarella
\end{itemize}}
{\begin{enumerate}
	\item Laver et hacher finement l'oignon, la carotte et la branche de céleri (en ayant pris soin d'en retirer les feuilles que l'on réserve pour plus tard).
	\item Dans une marmite ou une cocotte, verser un fond d'huile d'olive et y ajouter la moitié du mélange préalablement préparé, puis y ajouter toute la tomate.
	\item Ajouter ensuite un bouquet constitué des feuilles de céleri et de deux branches de basilic entières, bouquet que l'on retirera en fin de cuisson de la sauce
	\item Préparer une béchamel classique et la réserver.
	\item Prendre ensuite un autre récipient et y ajouter un fond d'huile d'olive, le reste de hachis (oignon + carotte + céleri), puis ajouter la chair à saucisse, laisser cuire environ 5 minutes et ajouter le boeuf.
	\item Lorsque le mélange a pris sa couleur, le retirer du feu et le mélanger avec la béchamel.
	\item Si vous avez le temps, laissez cuire la sauce tomate environ 1 heure à feux doux en rajoutant un peu d'eau si nécessaire.
	\item Procéder ensuite de la manière habituelle en faisant une couche de pâtes, une couche de tomate, une couche de béchamel + viande, en rajoutant quelques feuilles de basilic entre chaque couche.
	\item Laisser cuire environ 35 min à four moyen et recouvrir de mozzarella coupée en petits morceaux quelques minutes avant de servir.
\end{enumerate}}

\bigskip
\Recet{Lasagnes chèvre et épinard}{Préparation : 35 minutes \\ Cuisson : 30 minutes}
{Pour 6 personnes}{\begin{itemize}
	\item 1 kg d'épinards surgelés
	\item 400 g de fromage de chèvre (en bûche)
	\item 200 g de gruyère râpé
	\item 1 paquet de feuilles de lasagne
	\item 2/3 l de sauce béchamel
	\item sel et poivre
\end{itemize}}
{\begin{enumerate}
	\item Préchauffez le four à 200°C (Th 6-7).
	\item Décongelez les épinards, à feu doux, dans une casserole, puis enlevez l'eau résiduelle (n'hésitez pas à appuyer pour la faire sortir!).
	\item Hachez-les grossièrement.
	\item Emiettez la bûchette de chèvre.
	\item Dans un grand plat à four beurré, posez une couche de lasagnes, puis une couche d'épinards, puis une couche de miettes de chèvre, puis un peu de béchamel, puis un peu de sel et de poivre, puis à nouveau une couche de pâtes...
	\item Faites ainsi 2 ou 3 couches, en terminant par la béchamel et en recouvrant de gruyère râpé.
	\item Laissez cuire environ 30 min et servez chaud.
\end{enumerate}}

