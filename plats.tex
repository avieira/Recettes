\Recet{Penne aux courgettes}{Préparation : 20 minutes\\Cuisson : 10 minutes}
{Pour 1 personne}{\begin{itemize}
	\item 50 g de pâtes (penne ou autre)
	\item 1 courgette
	\item 1 petite boîte de tomates pelées (ou en morceaux)
	\item origan
	\item thym
	\item 1 feuille de laurier
	\item 2 cuillères à café d'huile
	\item 1 cuillère à soupe de parmesan
	\item sel, poivre
\end{itemize}}
{\phantom{.}

\bigskip
\begin{enumerate}
	\item Mettez à cuire les pâtes.
	\item Lavez et coupez la courgette en dés. Faire suer la courgette dans l'huile un instant, puis ajoutez 1 pincée d'origan, de thym et la feuille de laurier. Versez les tomates coupées en dés. Laissez cuire jusqu'à ce que les courgettes soient cuites. Assaisonner un peu selon votre goût.
	\item Egouttez les pâtes et disposez les dans une jolie assiette. Versez les courgettes dessus. Saupoudrez de parmesan.
\end{enumerate}

\bigskip
\phantom{.}}

\bigskip
\Recet{Poulet à la vietnamienne}{Préparation : 20 minutes \\ Cuisson : 15 minutes}
{Pour 4 personnes}{\begin{itemize}
	\item 800 g d'escalopes de poulet
	\item 2 oignons
	\item 1 verre de vin blanc
	\item sauce au soja
	\item gingembre moulu
	\item sel et poivre
\end{itemize}}
{\phantom{.}

\bigskip
\begin{enumerate}
	\item Découper en lanières le poulet, les saupoudrer de gingembre.
	\item Faire revenir les oignons émincés dans une sauteuse. Lorsque ceux-ci sont transparents, ajouter le poulet 
	\item Faire revenir 2 mn puis ajouter 2 cuillères de sauce au soja, le vin, sel et poivre et laisser cuire à couvert 10 mn.
\end{enumerate}

\bigskip
\phantom{.}}

\bigskip
\Recet{Filet mignon poêlé au vinaigre balsamique}{Préparation : 5 minutes \\ Cuisson : 30 minutes}
{Pour 5 personnes}{\begin{itemize}
	\item 1 filet mignon
	\item vinaigre balsamique
	\item miel
	\item crème fraiche liquide
	\item sel/poivre
	\item 1 noisette de beurre
\end{itemize}}
{\phantom{.}

\bigskip
\begin{enumerate}
	\item Faites revenir le filet mignon dans une poêle, avec 1 noisette de beurre.
	\item Ajouter le vinaigre balsamique, environ 20 ml, et le miel (3 bonnes cuillères à soupe). Saler et poivrer.
	\item Dès que la cuisson est bonne, retirer le filet mignon de la poêle, allonger la sauce avec de la crème fraiche liquide, et la servir pour accompagner votre filet mignon.
\end{enumerate}

\bigskip
\phantom{.}}

\bigskip
\Recet{Aubergines au chèvre chaud}{Préparation : 20 minutes\\Cuisson : 30 minutes}
{Pour 4 personnes}{\begin{itemize}
	\item 2 aubergines
	\item 3 petits fromages de chèvre mi-secs
	\item 6 cuil. à soupe d'huile d'olive
	\item 1 pincée de piment d'Espelette
	\item sel, poivre
\end{itemize}}
{\phantom{.}

\bigskip
\begin{enumerate}
	\item Epluchez les aubergines en laissant une bande de peau sur deux. Coupez-les en 18 tranches. Faites-les cuire 6 min à la poêle dans l'huile d'olive. Salez, poivrez.
	\item Superposez les tranches 3 par 3 en y intercalant les fromages de chèvre coupés en fines lamelles. Déposez-les dans un plat à rôtir. Enfournez 5 min th. 6/7 (200 °C). Relevez de piment d'Espelette. Servez sans attendre.
\end{enumerate}

\bigskip
\phantom{.}}

\bigskip
\Recet{Ratatouille niçoise}{Préparation : 30 minutes \\ Cuisson : 1h15}
{Pour 6 personnes}{\begin{itemize}
	\item 1kg de tomates
	\item 3 poivrons
	\item 500g de courgettes
	\item 2 aubergines
	\item 3 oignons
	\item 3 gousses d'ail
	\item 1 feuille de laurier
	\item 1/2 verre d'huile d'olive
	\item 1 pointe de cayenne
	\item 1 clou de girofle
	\item sel, poivre
\end{itemize}}
{\phantom{.}

\bigskip
\begin{enumerate}
	\item Coupez les poivrons en lamelles
	\item Épluchez et coupez les aubergines en dés
	\item De même avec les courgettes
	\item Émincez les oignons
	\item Dans un grand fait-tout, faîtes chauffer l'huile et mettez-y les oignons, les dés de courgettes et d'aubergines. Lorsqu'ils sont dorés, retirez-les et reservez-les.
	\item Dans le même fait-tout, faîtes également revenir l'ail et les poivrons. Au bout de 5 minutes, mettez les tomates. Laissez revenir une dizaine de minutes en remuant, puis ajoutez les courgettes et les aubergines reservées. Assaisonnez (pensez au clou de girofle !) et laissez cuire à couvert et à feu doux pendant 1 heure.
\end{enumerate}}

\bigskip
\Recet{Chili con carne}{Préparation : 1h30 \\ Cuisson : 2h}
{Pour 6 personnes}{\begin{itemize}
	\item 1 kg de paleron (viande qui devient plus moelleuse après une longue cuisson)
	\item 1 kg d'oignon frais
	\item 1 boîte de 500 g de haricots rouges sec
	\item 3 piments oiseaux
	\item 1 grosse boîte de maïs
	\item 2 boîtes de tomates pelées
	\item Clous de Girofle
	\item Huile d'olive
	\item vin blanc sec
\end{itemize}}
{\begin{enumerate}
	\item Faire tremper les haricots pendant au moins une nuit. Une fois le trempage effectué, rincer 3 fois minimum les haricots.
	\item Faire cuire les haricots (avec un gros oignon piqué de 2 clous de girofle) jusqu'à ce qu'ils aient la consistance des châtaignes cuites. Ils ne doivent surtout pas éclater car ils cuiront 2 bonnes heures par la suite. Une fois cuits, mettez-les de côté.
	\item Faire revenir les piments dans de l'huile d'olive. Il faut que les piments soient presque cramés. Je dis bien presque ! Jetez les piments : seule l'huile nous intéresse.
	\item Faites revenir la viande à feu très vif dans cette huile.
	\item Faites ensuite revenir les oignons dans cette même huile (ce qu'il reste, mais vous pouvez en ajouter un peu si vous voulez). Ils faut qu'ils soient bien dorés. (Vous pouvez ajouter un peu de vin blanc sec de temps en temps si vous le souhaitez)
	\item Quand vos oignons ont fini de cuire dans la sauce de la viande (qui a elle même cuit dans l'huile des piments), vous jetez dedans le contenu de 2 boîtes de tomates pelées. Si vous aimez vraiment la tomate, vous pouvez sans crainte ajouter 1 boîte de concentré. Laissez réduire.
	\item Quand vous trouvez que tout ça c'est pas mauvais du tout, ajoutez la viande et les haricots.
	\item Vous laissez mijoter assez longtemps (2h minimum, en principe). Quand vous sentez que c'est presque cuit, ajoutez une grosse boîte de maïs. Laissez cuire encore 10 min.
\end{enumerate}}

\bigskip
\Recet{B\oe uf bourgignon}{Prévoir au moins 3h de macération. Cuisson : au moins 4h (possibilité de commencer la veille !)}
{Pour 4 personnes}{\begin{itemize}
	\item 1kg de paleron 
	\item 50g de beurre 
	\item 1 petit verre de cognac
	\item 50cl de bourgogne rouge
	\item 40cl de bouillon de boeuf
	\item 1 bouquet garni
	\item 2 cuillerées à soupe de farine
	\item 200g de lardons
	\item 24 oignons grelots
	\item 400g de champignons de paris 
	\item sel, poivre
\end{itemize}}
{\begin{enumerate}
	\item Préparer le paleron. Enlever les parties dures, puis le couper en morceaux. Placer la viande dans un récipient, ajouter le verre de cognac et le vin rouge. Laisser macérer au moins trois heures ou alors une nuit au frais.
	\item Lorsque la viande est prête, bien l'essorer dans du papier absorbant, en prenant soin de garder la marinade. Faire fondre 25g de beurre dans une cocotte qui pourra aller au four. Faire dorer la viande sur feu très vif. Laisser cuire jusqu'à ce que le jus s'évapore et que la viande se colore.
	\item Enlever la viande de la cocotte. Puis y ajouter les 25g de beurre restant et la farine. Laisser dorer sur feu un peu plus doux.
	\item Quand le tout a une joli couleur dorée, ajouter la marinade, puis le bouillon de boeuf. On peut ne pas mettre le bouillon et ne mettre que du vin comme on le voit souvent, mais je préfère mon boeuf bourguignon préparé de cette façon. 
	\item Porter à ébullition. Lorsque la sauce bout, remettre la viande dans la cocotte. Saler et poivrer. On peut ajouter également les pieds des champignons de paris, bien nettoyés, et enfin le bouquet garni.
	\item Fermer le couvercle puis mettre à four doux (150 degrés) pendant au moins deux heures. On pourra arrêter à ce moment la cuisson si l'on décide de poursuivre le plat le lendemain.
	\item Une demie heure avant les deux heures de cuisson (ou alors le lendemain), préparer la suite. Faire dorer les lardons dans une poêle bien chaude. Quand ils sont bien dorés, les mettre de côté.
	\item Éplucher les oignons grelots et couper les champignons en tranches. Faire dorer les oignons dans le gras des lardons, dans la poêle bien chaude. Il suffit juste de les colorer, pas de les cuire.
	\item Au bout des deux heures, sortir la cocotte du four. Enlever les pieds des champignons et le bouquet garni, puis ajouter finalement les lardons et les oignons grelots. Bien mélanger puis ajouter les champignons coupés en tranches.
	\item Tout mélanger, reporter à ébullition, puis remettre au four pour une heure. 
	\item  Si la sauce est trop liquide au bout de ce temps de cuisson, recuire à découvert (sans le couvercle) sur feu doux ou ajouter un peu de farine. Il faut aussi vérifier la cuisson de la viande et des légumes et ajuster en fonction des saveurs et textures recherchées. 
	\item Servir bien chaud, avec par exemple des pâtes fraîches, et un bon vin rouge !
\end{enumerate}}

\bigskip
\Recet{Poulet à l'Estragon}{}
{Pour 4 personnes}{\begin{itemize}
	\item Un poulet coupé en morceaux
	\item 2 cuillerées à soupe d'huile d'olive
	\item 1 échalote
	\item 150ml de vin blanc sec
	\item 500ml de bouillon de poule
	\item 1 bouquet d'estragon
	\item 40cl de crème fleurette entière
	\item Sel, poivre
\end{itemize}}
{\begin{enumerate}
	\item Faites chauffer une cocotte sur feu modéré avec l'huile d'olive. Ajoutez le poulet. Laissez-le se colorer en le retournant de temps en temps. En tout cela prend une dizaine de minutes. 
	\item Ôtez le poulet de la cocotte en y laissant le gras. Hachez finement l'échalote et mettez à la place l'échalote. Remuez régulièrement pendant deux ou trois minutes puis ajoutez le vin et le bouillon de poule.
	\item Remettez le poulet. J'enlève à ce moment la plupart de la peau, sinon le plat devient trop gras. J'en laisse sur un morceau de haut de cuisse par exemple. Faites cuire sur modéré pendant vingt à trente minutes. Vérifiez bien la cuisson du poulet. Quand il est cuit, enlevez-le du bouillon et mettez-le de côté dans une assiette.
	\item Laissez maintenant réduire le bouillon de moitié. Cela peut prendre un peu de temps.
	\item Hachez grossièrement l'estragon. Mettez-le dans le bouillon, puis versez la crème entière. Salez et poivrez à votre goût. Laissez de nouveau réduire jusqu'à la consistance désirée.
	\item Quand c'est le cas, remettez le poulet et laissez réchauffer une dizaine de minutes avant de servir avec un riz nature, des pommes de terre vapeur ou des pâtes.
\end{enumerate}}

\bigskip
\Recet{Risotto au poulet}{Préparation : 35 minutes\\Cuisson : 1 heure}
{Pour 6 personnes}{\begin{itemize}
	\item 500 g de riz Arborio
	\item 3 blancs de poulet
	\item 200 g de lardons fumés
	\item 200 g de petits pois
	\item 15 cl de vin blanc sec
	\item 1,5 l de bouillon de volaille
	\item 10 cl d'huile d'olive
	\item 20 g de beurre
	\item 20 cl de crème fraîche (facultatif)
	\item 4 échalotes
	\item 2 gousses d'ail
\end{itemize}}
{\begin{enumerate}
	\item Chauffer le bouillon de volaille (ou 1,5 l d'eau et 3 cubes). Cuire les petits pois 10 minutes à l'eau bouillante légèrement salée et additionnée d'une pincée de sucre. Ebouillanter les lardons 1 minute pour les dessaler.
	\item Pendant ce temps, éplucher et émincer l'ail et les échalotes. Découper le poulet en fins bâtonnets ou en lamelles.
	\item Chauffer l'huile et le beurre dans une sauteuse et y faire revenir les lamelles de poulet et les lardons 2 minutes. Les retirer et réserver au chaud. Ajouter dans la sauteuse l'ail et les échalotes. Faire revenir 1 minute sans les brûler.
	\item Ajouter le riz en remuant à la cuillère en bois pour bien imprégner chaque grain de matière grasse. Déglacer avec le vin blanc. Porter à ébullition. Ajouter une louche de bouillon de volaille bien chaud tout en remuant. Lorsque cette louche de bouillon est absorbée par le riz, en ajouter une autre tout en remuant et ainsi de suite jusqu'à utilisation de tout le bouillon. Avec la dernière louche de bouillon, ajouter le poulet, les lardons et les petits pois égouttés.
	\item Juste avant de servir, ajouter la crème fraîche tout en continuant à remuer. 
\end{enumerate}}

\bigskip
\Recet{Risotto aux Morilles}{}
{Pour 4 personnes}{\begin{itemize}
	\item 300g de riz carnoli ou arborio
	\item 1 litre de bouillon de légumes
	\item 25g de morilles sèches
	\item 35g de parmesan
	\item Un verre de vin blanc
	\item 2 petites échalottes
	\item 30g de beurre et une cuillerée à soupe d'huile d'olive
	\item Ciboulette
\end{itemize}}
{\begin{enumerate}
	\item Si on utilise des morilles séchées, les mettre dans une jatte et y verser de l'eau tiède/chaude. Laisser gonfler les morilles pendant un bon quart d'heure, garder l'eau de trempage pour cuire le riz avec. Le riz aura en plus le parfum des morilles dans sa cuisson !
	\item Rajouter de l'eau pour obtenir 1 litre de bouillon et rajouter à cela 2 cubes de bouillon de légumes (ou rajouter directement votre propre bouillon de légume à l'eau de trempage !)
	\item Cuire les morilles trempées à l'eau bouillante pendant 10 minutes.
	\item Bien égoutter et couper les morilles en morceaux. Mettre 15g de beurre (sur les 30) dans une poêle et faire revenir les morilles à feu doux. Saler et poivrer légérement. Une fois que les morilles commencent à dorer, arrêter le feu et mettre de côté.
	\item Mettre les 15g de beurre restant et l'huile d'olive dans une casserole, et faire revenir les échaltottes finement coupées à feu doux. Ne pas laisser dorer, juste se développer les arômes. 
	\item Ajouter le riz. Bien mélanger pour bien enrober tous les grains de matière grasse. Verser le vin blanc. Bien mélanger, le riz va alors rapidement absorber tout le liquide. Bien mélanger toujours à feu doux.
	\item Ajouter une première grosse louchée de bouillon aux morilles. Bien mélanger et laisser le riz absorber le liquide. Puis remettre une autre louchée de bouillon, laisser absorber, et ainsi de suite. Mélanger à la cuillère de temps en temps pour éviter au fond d'attacher.
	\item Pendant que le risotto mijote, râper le parmesan frais. 
	\item Une fois tout le bouillon bu par le riz, verser le parmesan. On va maintenant "mantecare" ! Ce qui veut dire que l'on va beurrer le ristotto! Mais là, la matière grasse sera le parmesan. Bien mélanger vivement le risotto à ce moment pour bien incorporer tout le fromage. Cette opération se fait normalement avec une cuiller en bois avec un trou dedans.
	\item Ajouter les morilles tout en mélangeant. Puis pour finir, la ciboulette ciselée. 
	\item Servir aussitôt car le risotto n'attend pas! Il se déguste bien chaud.
\end{enumerate}}

\bigskip
\Recet{Risotto aux épinards}{}
{Pour 4 personnes}{\begin{itemize}
	\item 300g de riz carnoli ou arborio
	\item 1 litre de bouillon de légumes
	\item 1 oignon
	\item 15cl de vin blanc doux
	\item 2 cuillerées à soupe d'huile d'olive
	\item 40g de beurre salé
	\item 500g d'épinards
	\item 125g de mascarpone
	\item 40g de parmesan frais râpé
\end{itemize}}
{\begin{enumerate}
	\item Laver les feuilles d'épinards avec précaution. Enlever la nervure centrale des feuilles. Bien essorer.
	\item Faire fondre le beurre salé dans une sauteuse. Ajouter au fur et à mesure les feuilles d'épinards. Il y a un grand volume qui va considérablement réduire en cuisant.
	\item Quand l'eau que les épinards ont rendu s'est évaporée (un bon quart d'heure de cuisson sur feu assez fort), ajouter le mascarpone.  Bien mélanger, saler et poivrer...goûter!
	\item Laisser cuire un peu plus, jusqu'à ce que le mascarpone soit "absorbé" par les épinards. En prenant une cuillerée d'épinards, on voit si la crème s'évade ou si les épinards et la crème forment un tout.  À ce moment, c'est cuit ! Arrêter la cuisson et mettre de côté.
	\item Préparer le riz pour le risotto comme pour la recette du risotto aux morilles (en remplaçant les échalottes par l'oignon)
	\item Au moment de mettre normalement les morilles, rajouter les épinards par étape, jusqu'à obtenir la consistance souhaitée. Ajouter le parmesan fraichement râpé. 
	\item Bien mélanger et servir immédiatement avec un peu de parmesan râpé !
\end{enumerate}}

\bigskip
\Recet{Lasagnes}{Préparation : 1h30 \\ Cuisson : 35 minutes}
{Pour 6 personnes}{\begin{itemize}
	\item 1/2 paquet de pâte à lasagnes
	\item 250 g de viande hachée de boeuf
	\item 150 g de chair a saucisse
	\item 1 boîte de tomates en dés
	\item 1 bouteille de purée de tomate (passata)
	\item 1 oignon
	\item 1 carotte
	\item 1 branche de céleri
	\item huile d'olive
	\item sel, poivre
	\item basilic (frais si possible)
	\item mozzarella
\end{itemize}}
{\begin{enumerate}
	\item Laver et hacher finement l'oignon, la carotte et la branche de céleri (en ayant pris soin d'en retirer les feuilles que l'on réserve pour plus tard).
	\item Dans une marmite ou une cocotte, verser un fond d'huile d'olive et y ajouter la moitié du mélange préalablement préparé, puis y ajouter toute la tomate.
	\item Ajouter ensuite un bouquet constitué des feuilles de céleri et de deux branches de basilic entières, bouquet que l'on retirera en fin de cuisson de la sauce
	\item Préparer une béchamel classique et la réserver.
	\item Prendre ensuite un autre récipient et y ajouter un fond d'huile d'olive, le reste de hachis (oignon + carotte + céleri), puis ajouter la chair à saucisse, laisser cuire environ 5 minutes et ajouter le boeuf.
	\item Lorsque le mélange a pris sa couleur, le retirer du feu et le mélanger avec la béchamel.
	\item Si vous avez le temps, laissez cuire la sauce tomate environ 1 heure à feux doux en rajoutant un peu d'eau si nécessaire.
	\item Procéder ensuite de la manière habituelle en faisant une couche de pâtes, une couche de tomate, une couche de béchamel + viande, en rajoutant quelques feuilles de basilic entre chaque couche.
	\item Laisser cuire environ 35 min à four moyen et recouvrir de mozzarella coupée en petits morceaux quelques minutes avant de servir.
\end{enumerate}}

\bigskip
\Recet{Lasagnes chèvre et épinard}{Préparation : 35 minutes \\ Cuisson : 30 minutes}
{Pour 6 personnes}{\begin{itemize}
	\item 1 kg d'épinards surgelés
	\item 400 g de fromage de chèvre (en bûche)
	\item 200 g de gruyère râpé
	\item 1 paquet de feuilles de lasagne
	\item 2/3 l de sauce béchamel
	\item sel et poivre
\end{itemize}}
{\begin{enumerate}
	\item Préchauffez le four à 200°C (Th 6-7).
	\item Décongelez les épinards, à feu doux, dans une casserole, puis enlevez l'eau résiduelle (n'hésitez pas à appuyer pour la faire sortir!).
	\item Hachez-les grossièrement.
	\item Emiettez la bûchette de chèvre.
	\item Dans un grand plat à four beurré, posez une couche de lasagnes, puis une couche d'épinards, puis une couche de miettes de chèvre, puis un peu de béchamel, puis un peu de sel et de poivre, puis à nouveau une couche de pâtes...
	\item Faites ainsi 2 ou 3 couches, en terminant par la béchamel et en recouvrant de gruyère râpé.
	\item Laissez cuire environ 30 min et servez chaud.
\end{enumerate}}

\bigskip
\Recet{Lasagnes aux aubergines}{}
{Pour 6 personnes}{\begin{itemize}
	\item 3 aubergines
	\item 8 tomates
	\item 2 bonnes cuillerées à soupe de concentré de tomates
	\item 4 oignons
	\item un bouquet de basilic
	\item un bouquet garni
	\item un paquet de lasagnes
	\item 100g de parmesan râpé
	\item 300 à 400g de mozzarelle
	\item 40cl de crème liquide
	\item sel, poivre
	\item huile d'olive
\end{itemize}}
{\begin{enumerate}
	\item Peler les tomates : les ébouillanter quelques secondes, et retirer la peau.
	\item Couper les oignons en morceaux. Mettre à cuire dans une poêle avec de l'huile d'olive sur feu modéré. 
	\item Quand ils sont translucides, ajouter les tomates pelées et coupées en gros morceaux. Laisser cuire en mélangeant de temps en temps, ajouter le bouquet garni et le concentré de tomates. 
	\item Quand le tout parait bien cuit (une vingtaine de minutes), ajouter la crème. Couper le basilic en morceaux avec un couteau (ou une paire de ciseaux), et ajouter dans la crème de tomates. Bien mélanger, couper le feu, saler et poivrer et mettre de côté en attendant la suite de la recette.
	\item Couper les aubergines en tranches d'un centimètre d'épaisseur. On peut ensuite les passer 7 minutes au micro-ondes pour les précuire. Si l'on ne possède pas ce four, on saute cette étape, mais la cuisson à la poêle sera plus longue! 
	\item Mettre les aubergines dans une poêle assez chaude avec de l'huile d'olive. Laisser cuire jusqu'à ce que les aubergines soient bien dorées, puis les retourner. Quand elles sont bien cuites et dorées, les laisser refroidir sur du papier absorbant, puis saler et poivrer chaque côté des aubergines. Il faut pendant ce temps, continuer de cuire les aubergines suivantes!
	\item Placer des lasagnes dans un plat. Ajouter un tiers de la crème à la tomate. Parsemer de parmesan râpé. Ajouter une couche d'aubergines.  Ajouter de nouveau des lasagnes. Ajouter ensuite un nouveau tiers de sauce tomates.  Ajouter le reste des aubergines, saupoudrer de parmesan et finir avec le dernier tiers de sauce. 
	\item  Ajouter la mozzarelle coupée en tranches. Saler, poivrer et arroser légèrement d'huile d'olive.  Cuire à 200°C pendant 35 minutes. 
\end{enumerate}}

\bigskip
\Recet{Gambas flambées}{}
{Pour 4 personnes}{\begin{itemize}
	\item 20 grosses gambas
	\item 4 ou 5 gousses d'ail
	\item 1/2 bouquet de persil haché très finement
	\item 1 oignon
	\item 40g de beurre salé
	\item 10cl de porto
	\item sel, poivre
\end{itemize}}
{\begin{enumerate}
	\item Commencer par éplucher l'ail et l'oignon. Couper l'oignon en morceau et l'ail en tranches. Mettre le beurre sur feu modéré. Quand il rissole, ajouter l'ail. Laisser colorer l'ail légèrement et le laisser développer son parfum. Ajouter l'oignon et le persil.
	\item Bien mélanger puis poser les gambas (sans la tête mais avec leur carapace) sur le lit d'ail, d'oignon de persil. Augmenter légèrement le feu. Retourner les gambas pour cuire l'autre face.
	\item Chauffer le porto dans une petite casserole. Flamber le porto puis le verser sur les gambas. 
	\item Bien mélanger pour fondre les sucs au fond de la poêle. Servir les gambas immédiatement telles quelles ou accompagnées de riz !
\end{enumerate}}

\bigskip
\Recet{Brocolis sautés aux grosses crevettes et piment}{Préparation : 30 minutes \\ Cuisson : 20 minutes}
{Pour 4 personnes}{\begin{itemize}
	\item 400g de brocolis
	\item Une vingtaine de crevettes
	\item 2 gousses d'ail
	\item 20g de gingembre frais
	\item le jus d'un citron vert
	\item 5 cuill. à soupe d'olive
	\item 1 piment oiseau frais
	\item Sel et poivre du moulin
\end{itemize}}
{\begin{enumerate}
	\item Séparez les têtes des queues des crevettes, puis décortiquez-les. Pelez et pressez les gousses d'ail et le gingembre. Arrosez les crevettes de jus de citron vert, puis parsemez-les de gingembre et d'ail. Ajoutez 1 cuillerée à soupe d'huile, salez, poivrez, puis mélangez délicatement et laissez mariner au réfrigérateur.
	\item Remplissez un grand saladier d'eau froide et de glaçons et portez une casserole d'eau salée à ébullition. Coupez les queues des brocolis, détachez les petits bouquets et lavez-les sous l'eau fraîche puis metez-les à cuire dans l'eau bouillante 3 minutes. Égouttez-les puis plongez-les immédiatement dans le saladier d'eau glacée pour arrêter la cuisson. Patientez 5 minutes, puis égouttez-les soigneusement sur un linge.
	\item Lavez, équeutez, épépinez et hachez finement le piment.
	\item Faîtes chauffer 3 cuillerées à soupe d'huile d'olive dans une poêle, puis ajoutez les crevettes et leur marinade. Faites cuire 4 minutes à feu vif, puis ajoutez les petits bouquets de brocolis, le piment et poursuivez la cuisson 4 minutes en mélangeant délicatement tous les ingrédients.
\end{enumerate}}

\bigskip
\Recet{Gratin de courgettes aux amandes}{Préparation : 20 minutes \\ Cuisson : 35 minutes}
{Pour 4 personnes}{\begin{itemize}
	\item 4 belles courgettes bien fermes
	\item 3 cuill. à soupe d'huile d'olive
	\item 15cl de crème liquide
	\item 2 \oe ufs + 1 jaune
	\item 1 pointe de noix de muscade rapée
	\item 125g de parmesan râpé
	\item 75g d'amandes en poudre
	\item 10g de beurre
	\item 30g d'amandes effilées
	\item Sel et poivre du moulin
\end{itemize}}
{\begin{enumerate}
	\item Lavez, puis coupez-les les courgettes en rondelles de 5 mm d'épaisseur.Faîtes chauffer l'huile d'olive dans une poêle, puis faîtes dorer les rondelles de courgettes 2 minutes de chaque côté. Salez, poivrez, puis laissez-les refroidir en les posant sur une feuille de papier absorbant.
	\item Fouettez la crème liquide avec les oeufs entiers et le jaune. Saupoudrz de noix de muscade d'un peu de sel et de poivre, du tiers du parmesan râpée et des amandes en poudre. 
	\item Préchauffez le four à 180°C (therm. 6)
	\item Beurrez un plat à gratin. Déposez une couche de rondelles de courgettes, en les faisant se chevaucher, puis couvrez du mélange à la crème. Disposez une seconde couche de courgettes, puis de crème, et ainsi de suite jusqu'à épuisement des ingrédients. Finissez par une couche de crème. Parsemez d'amandes effilées et du parmesant restant, enfournez et faîtescuire 30 minutes.
	\item Lorsque le gratin est bien doré, sortez-le du four et servez-le aussitôt !
\end{enumerate}}

\bigskip
\Recet{Rouleaux de printemps aux crevettes, germes de soja}{Préparation : 35 minutes \\ Cuisson : 20 minutes}
{Pour 4 personnes}{Pour les rouleaux : \begin{itemize}
	\item 100g de germes de soja
	\item 20g de gingembre frais
	\item 50cl de bouillon de volaille
	\item 1 petite escalope de poulet
	\item 1 petite poignée de vermicelles de riz
	\item 4 grandes feuilles de riz
	\item 1 carotte bien tendre
	\item 8 queues de crevettes cuites décortiquées
	\item 4 petites feuilles de laitue
	\item 12 feuilles de menthe (facultatif)
	\item Sel et poivre du moulin
\end{itemize}
Pour la sauce : \begin{itemize}
	\item 1 petite gousse d'ail finement hachée
	\item 1 petit piment épépiné et finement haché
	\item 5 cuill à soupe de sauce de soja sucrée
	\item 2 cuill. à soupe de bouillon de volaille
\end{itemize}}
{\phantom{.}

\bigskip
\begin{enumerate}
	\item Pelez et écrasez le gingembre, puis plongez-le dans le bouillon de volaille instantané. Faîtes chauffer jusqu'à frémissement, puis ajoutez l'escalope de poulet et prolongez la cuisson 15 minutes. Égouttez le poulet et laissez-le refroidir.
	\item Portez à nouveau le bouillon à ébullition et faites blanchir les germes de soja 5 secondes pour les attendrir, puis égouttez-les sur du papier absorbant. Dans ce même bouillon, ébouillantez les vermicelles de riz 2 minutes, puis égouttez-les et laissez-les refroidir.
	\item Posez une galette de riz sur un plan de travail humide. Repliez un quart de cette galette, puis disposez, sur ce quart doublé, la feuille de laitue, les germes de soja, un peu de poulet, de carotte, de vermicelles de riz et 3 feuilles de menthe. Rabattez les deux côtés de la galette et roulez-la comme un cigare. À mi-chemin, posez 2 semi-crevettes, puis continuez à rouler la feuille de riz. Procédez de la même manière pour les 3 autres galettes de riz, puis emballez chaque rouleau de printemps dans un film alimentaire et réservez au réfrigérateur. 
	\item Préparez la sauce en mélangeant tous les ingrédients. Servez les rouleaux de printemps avec quelques feuilles de romaine bien croquante et la sauce à part.
\end{enumerate}

\medskip
\phantom{.}}

\bigskip
\Recet{Flammeküche}{}
{Pour 2 flammeküches}{
 Pour la pâte:
\begin{itemize}
  \item 250g de farine
  \item 50ml d'huile
  \item une bonne pincée de sel
  \item de l'eau
\end{itemize}
Pour la garniture:
\begin{itemize}
  \item 90g de crème épaisse
  \item 90g de fromage blanc
  \item un gros oignon
  \item un demi verre de vin blanc sec
  \item 150g de lardons allumettes fumés (ou non si on préfère!)
\end{itemize}}
{\begin{enumerate}
	\item Commencer par préparer la pâte. Mettre la farine, le sel et l'huile dans un bol. Commencer à mélanger du bout des doigts. Puis ajouter suffisamment d'eau pour former une pâte. La quantité d'eau dépend du type de farine utilisé. Il faut y aller petit à petit. Puis pétrir la pâte soit à la machine avec un crochet à pâte, soit à la main pendant 10 minutes. Mettre cette boule de pâte dans un film étirable puis une heure au frais pour la détendre!
	\item Emincer l'oignon finement. Mettre une noix de beurre et une petite cuillerée d'huile dans une poêle et chauffer à feu doux. Ajouter l'oignon émincé. Attention : les oignons ne doivent pas colorer. Ajouter un demi verre de vin blanc et laisser cuire jusqu'à ce que tout le vin se soit évaporé. Eteindre le feu et réserver.
	\item Préchauffer le four sur 280°C. Couper la boule de pâte en deux. Aplatir une première moitié avec la paume de la main sur un papier sulfurisé. Puis avec un rouleau, étaler la pâte le plus finement possible.
	\item Avec un cercle à gâteau, couper un cercle de pâte. On peut couper simplement au couteau en essayant au mieux de faire un beau cercle! Plier tout le pourtour de la pâte en essayant de faire une frise. 
	\item Mélanger dans un bol la crème entière et le fromage blanc et étaler deux cuillerées à soupe de ce mélange sur la pâte. Bien étaler jusqu'au bord plié.
	\item Ajouter la moitié des lardons sur le dessus de la tarte. Puis la moitié des oignons. Saler et poivrer. Effectuer la même opération avec la deuxième boule de pâte pour faire une deuxième flammeküche.
	\item Passer au four pendant une dizaine de minutes ou jusqu'à ce que la tarte ait l'air cuite ! Il faut surveiller la cuisson de près car la pâte étant fine, elle brûle facilement.
\end{enumerate}
}
