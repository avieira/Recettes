\Recet{Pâtes aux \oe ufs}{Préparation : 30 minutes \\ Repos : $\sim$ 2 heures}
{}{\begin{itemize}
	\item 300g de farine
	\item 1 oeuf extra-frais (de poule élevée en plein air svp!)
	\item 4 jaunes d'oeufs
	\item 30ml d'huile d'olive
	\item de l'eau ($\sim$ 1/2 verre)
\end{itemize}}{\begin{enumerate}
	\item Commencer par mettre la farine dans une jatte, ou le bol du pétrisseur si on en a un. Creuser un puit au milieu et ajouter l'oeuf, les jaunes et l'huile d'olive.
	\item Commencer à pétrir soit à la main soit à la machine. Je commence avec la machine et je finis à la main car je trouve que la chaleur de la main joue beaucoup. Pétrir et ajouter juste ce qu'il faut d'eau. Il faut l'ajouter en filet et par petites quantités pour ne pas détremper la pâte si on en met trop d'un coup. Si il n'y en a pas assez, on le verra à la présence de farine sèche, si on en met trop on pourra rajouter un peu de farine pour rectifier. La farine nous dira d'elle même si elle a soif ou non ! Donc tout dépend d'elle et de son type (45, 55, 65).
	\item Rouler la pâte en boudin, plier en trois, écraser la pâte de la paume de la main, tourner la pâte de 90° puis rouler de nouveau en boudin et ainsi de suite. Procéder de cette façon pendant au moins dix minutes. La pâte va finir par s'attendrir et avoir une texture très douce. A ce moment la pâte est prête! La ramener en boule, fariner-la, et enfermer-la dans un film étirable. Placer cette boule de pâte au réfrigérateur pendant au moins une heure. La pâte va se détendre et plus facile à laminer.
	\item On peut réaliser des pâtes sans laminoir mais le résultat sera plus long à obtenir. Je préfère donc largement l'emploi de celui-ci. Fariner le plan de travail, couper la pâte en quatre, puis écraser chaque partie de la paume de la main. Procéder ensuite au laminage en commencant bien sûr toujours par le premier cran. Si on désire obtenir de belles bandes de pâtes (pour les lasagnes), il faut laminer une première fois, plier la bande obtenue en trois, la repasser au laminoir, replier en trois et cette fois tourner de 90°. La bande sera ainsi plus régulière et bien droite sur toute sa longueur (comme sur la photo). Sinon peu importe car la pâtes sera débitée en tagliatelle ou linguine. Passer à chaque deux ou trois fois sur chaque cran chaque bande de pâte. Au deuxième cran, couper les bandes en deux car sinon les bandes deviennent trop longues. Sauf bien sûr si c'est un choix!
	\item Laminer jusqu'au cran désiré. Laisser ensuite sécher les pâtes sur un séchoir à pâtes. Suivant l'hygrométrie de la pièce, sécher entre 15 minutes ou une heure trente ! On ne veut pas qu'elles soient complétement sèches, sinon on ne pourra pas les couper en tagliatelles ou autres. Sauf si on veut des lasagnes encore une fois !
	\item Passer ensuite au coupe pâtes.Laisser ensuite sécher les pâtes obenues de nouveau sur le séchoir à pâtes. Cette fois c'est au goût de chacun. Je préfère bien laisser sécher. Plus on laisse sécher, plus la cuisson des pâtes sera longue. Je ne peux donner aucune indication de cuisson. Tout dépend de l'humidité interne des pâtes, de l'épaisseur de celles-ci et de la cuisson souhaitée! Cela va de quelques secondes à plus de dix minutes!!! Le mieux est de goûter les pâtes en cours de cuisson. Cuire dans une grande quantité d'eau salée (mais sans huile!!) et adapter avec les recettes choisies! Sinon conserver les pâtes au frais dans une boîte hermétique. en les farinant bien avant de les ranger.
\end{enumerate}}

\bigskip
\Recet{Pâtes à pizza}{}
{}{Poolish:
\begin{itemize}
	\item 300g de farine T55
 	\item 300g d'eau tiède
	\item 1 petite cuillerée à café rase de levure instantanée de boulangerie ou 2,5g de levure fraîche
\end{itemize}

Pâte:
\begin{itemize}
	\item la poolish de la veille
	\item 300g de farine
	\item 30g d'eau tiède
	\item 10g de sucre
	\item 10g de sel
	\item 10g d'huile d'olive
	\item 1/2 cuillerée à café de levure instantanée ou 2g de levure fraiche
\end{itemize}}
{\begin{enumerate}
	\item Commencez par préparer la poolish la veille: cela est bien plus simple qu'il n'y parait. Il suffit de diluer la levure dans l'eau. Mettez la farine et l'eau/levure dans une cuve. Mélangez avec le "k" du robot ou avec un fouet. Il faut mélanger ainsi pendant trois ou quatre minutes. Cela donne une sorte de pâte liquide. 
	\item Racler la pâte pour la faire retomber dans la masse pour éviter qu'elle ne sèche durant la nuit. Laissez ensuite reposer au réfrigérateur toute la nuit jusqu'au lendemain matin avec un film étirable à la surface de la cuve. Vous pouvez aussi laisser monter la poolish pendant 4 à 5 heures en dehors du frigo. Mais au froid, elle montera plus lentement et pourra même se garder jusqu'à 48h.
	\item Le lendemain la poolish a doublé de volume et est pleine de bulles. Préparez maintenant votre pâte à pizza en ajoutant le reste des ingrédients (huile, eau, levure, farine, sel, sucre) avec la poolish. Essayez de mettre le sel et le sucre sur la farine et pas dans la poolish.
	\item Pétrissez cette fois avec le crochet. Laissez ainsi pétrir pendant 10 à 15 minutes. La pâte va devenir douce d'apparence et un peu collante au toucher. Filmez la pâte au contact. Puis laissez lever 1h30 à 2h00.
	\item Pour travailler, farinez bien vos mains car la pâte colle un peu et est bien molle. Prenez la pâte et sortez une boule de la taille de votre poing en l'écrasant entre l'index et le pouce. Farinez bien cette boule de pâte. Étendre la pâte en un cercle régulier. Mettre les ingrédients dessus, et enformer pendant 8 minutes à 250°C ! Surveillez bien la cuisson !
\end{enumerate}}

\bigskip
\Recet{Pâte feuilletée}{}
{}{Pour la détrempe:
\begin{itemize}
	\item 500g de farine
	\item 150g de beurre
	\item 12g de sel
	\item 250ml d'eau
\end{itemize}

Pour la finalisation:
\begin{itemize}
	\item 400g de beurre
\end{itemize}}
{\begin{enumerate}
	\item Commencer par préparer la détrempe. Mettre la farine et le beurre dans un bol. Sabler le tout.
	\item Dissoudre le seil dans l'eau. Ajouter l'eau salée en une seule fois puis mélanger avec le "k" ou à la main juste pour homogénéiser le tout sans travailler plus longtemps la pâte. La pâte est assez molle à ce stade et légèrement collante. 
	\item Mettre la détrempe sur un film étirable en essayer de lui une forme rectangulaire de 2 ou 3 centimètres d'épaisseur. Laisser reposer la détrempe au réfrigérateur pour environ 2 heures. 
	\item Fariner légèrement le plan de travail. Il ne faut pas mettre trop de farine au risque de changer les proportions de la pâte ! Etaler la pâte de façon à ce qu'elle soit 3 fois plus longue que large.  
	\item Mettre le beurre au centre en forme de carré, en laissant donc un tiers de détrempe à nue en haut en un tiers en bas. Le beurre doit être de la même consistance que la détrempe !
	\item Rabattre le tiers du bas sur le beurre puis le tiers supérieur sur la pâte qui couvre donc le beurre. Tourner la pâte de 90°. On se retrouve donc avec la pliure sur la droite.
	\item Etaler la pâte en commençant toujours par  appuyer légèrement au milieu du pâton, de façon à bien régulariser le tout. Fariner au besoin légèrement en caressant la pâte pour appliquer une couche fine de farine... Etaler la pâte de façon à ce qu'elle soit de nouveau 3 fois plus longue que large. 
	\item Plier en trois comme précédemment et tourner de 90°. Ceci est un tour simple. On pourrait étaler la pâte de façon à ce qu'elle soit 4 fois plus longue que large et plier la pâte en quatre. Ces tours sont dans ce cas des tours doubles.
	\item Recommencer deux fois cette opération pour avoir un total de 3 tours simples ou pour avoir deux tours doubles.
	\item A ce stade, plier en trois et emballer dans du film étirable et remettre au frais pendant une heure ou deux. 
	\item Sortir la pâte du réfrigérateur et appliquer 3 nouveaux tours simples à la pâte ou deux tours doubles. Rabattre une dernière fois en trois et laisser reposer la pâte au réfrigérateur jusqu'à utilisation sous film étirable.
\end{enumerate}}
