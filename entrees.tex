\Recet{Naan au fromage}{Préparation : 30 minutes + 4 heures \\ Cuisson : 5 minutes}
{Pour 4 naans}{\begin{itemize}
	\item 300 g de farine de blé
	\item un yaourt nature
	\item fromage de chèvre
	\item 2 bonnes pincées de sucre
	\item 2 pincées de levure
	\item du sel à votre goût
	\item une cuillère à soupe d’huile
	\item 5 cl d’eau chaude
\end{itemize}}
{\begin{enumerate}
	\item Mettre 200g de farine dans un récipient, verser une cuillère à soupe d’huile et bien mélanger. Ajouter la levure, le sucre, le sel et le yaourt et mélanger. Ajouter l’eau et mélanger. Rajouter le reste de farine pour frome une boule plus lisse, puis enduire d’huile. Assouplir la pâte en la tapant. Quand la pâte est bien souple, la laisser reposer dans un endroit tiède pendant 3 ou 4 heures. 
	\item Pré-chauffer le four à 300°C pendant 15 mn.
	\item Quand la pâte est reposée, la diviser en quatre boules. Etaler les naans, mettre le fromage au milieu et fermer la pâte en triangle ou en rond. Etaler de nouveau les naans. 
	\item Mettre les naans sur la plaque et les mettre au four pendant environ 5 mn à 260°C.
\end{enumerate}}

\bigskip
\Recet{Samoussas au thon et au curry}{Préparation : 30 minutes \\ Cuisson : 10 minutes}
{Pour 12 samoussas}{\begin{itemize}
	\item 6 feuilles de brick
	\item une boite de thon
	\item du curry
	\item poivre gris
	\item de la sauce soja
	\item de la crème fraîche épaisse
	\item de l'emmental râpé
	\item un oignon
\end{itemize}}
{\begin{enumerate}
	\item Peler et faire revenir rapidement l'oignon a la poêle.
	\item Dans un saladier mélangez la boite de thon, environ 2 cuillères à soupe de crème fraîche épaisse, le curry à doser selon son goût, l'oignon, une poignée de fromage râpé, assaisonnez avec la sauce soja et le poivre moulu.
	\item Mixez votre mélange vulgairement juste pour que toutes les saveurs soient bien réparties.
	\item Prenez vos feuilles de brick, coupez les en 2, mettez la garniture (qui doit avoir une consistance assez épaisse quand même) et confectionnez vos pliages de samoussas : mettre la garniture dans un coin, plier pour former un triangle. Replier dans l'autre sens, et recommencer l'opération.
	\item Mettez dans une poêle de l'huile d'olive et faites cuire vos samoussas jusqu'à ce qu'ils soient bien dorés ( environ 2 min de chaque coté)
\end{enumerate}}

\bigskip
\Recet{Cake aux lardons et aux olives}{Préparation : 25 minutes \\ Cuisson : 45 minutes}
{}{\begin{itemize}
	\item 200 g de farine
	\item 1 sachet de levure
	\item 3 oeufs
	\item 1 pincée de sel
	\item poivre moulu
	\item 10 cl d'huile d'olive
	\item 10 cl de lait chaud
	\item 100 g de gruyère râpé
	\item 200 g de lardons
	\item 150 g d'olives vertes dénoyautées
\end{itemize}}
{\begin{enumerate}
	\item Faire bouillir de l'eau et y plonger les lardons 2 min.
	\item Les égoutter puis les faire revenir dans une poêle, sans matières grasses, jusqu'à ce qu ils soient bien dorés. Réserver.
	\item Dans un saladier, mettre la farine, la levure, les oeufs, le sel, le poivre et l'huile.
	\item Bien mélanger, puis ajouter le lait chaud, le fromage râpé, les olives et les lardons.
	\item Graisser un moule à cake et y verser la préparation.
	\item Mettre au four 45 min environ à thermostat 6/7 (200°C), en couvrant avec un papier aluminium à mi-cuisson si le cake dore trop vite.
\end{enumerate}}

\bigskip
\Recet{Tarte aux patates douces, amandes et fromage bleu}{}
{Pour ne tarte de 22cm}{Pour la pâte :
\begin{itemize}
	\item 250g de farine
	\item 10g de sucre
	\item 125g de beurre demi-sel froid
	\item eau froide
\end{itemize}
Pour la garniture
\begin{itemize}
	\item 1 patate douce
	\item huile d'olive
	\item 150g de fromage bleu (Bleu d'Auvergne, Fourme d'Ambert, Saint Agur etc.)
	\item 75g d'amandes entières
	\item 2 œufs
	\item 250ml de crème liquide entière
	\item sel, poivre, noix de muscade
\end{itemize}}
{\begin{enumerate}
	\item Mettez la farine, le sucre et le beurre froid coupé en morceaux dans un récipient ou la cuve d'un robot. Mélangez avec la feuille ou du bout des doigts pour obtenir une consistance sableuse.
	\item Versez juste assez d'eau froide pour que la pâte devienne homogène. Ne mélangez surtout pas trop sinon la pâte deviendra dure. Mettez dans un film étirable au frais pendant 30 minutes. 
	\item Étalez la pâte légèrement farinée sur un papier sulfurisé sur une épaisseur de 3mm. Garnissez-en un cercle ou un moule à tarte puis coupez l'excédent.
	\item Laissez reposer la pâte au frais pendant 45 minutes.
	\item Épluchez la patate douce, et coupez-la en gros cubes. Arrosez avec deux ou trois cuillerées à soupe d'huile d'olive et salez à votre goût. Mettez les morceaux à cuire à 190°C pendant une vingtaine de minutes au four.
	\item Laissez les morceaux tiédir puis placez-en à votre convenance dans la pâte. Ajoutez des amandes. Mettez des morceaux de fromage bleu.
	\item Mélangez à la fourchette la crème avec les œufs, salez, poivrez et mettez un peu de noix de muscade puis arrosez la tarte pour arriver à 1 mm du bord.
	\item Enfournez et laissez cuire à 180°C une quarantaine de minutes.
	\item Laissez tiédir, dégustez avec une bonne salade ! 
\end{enumerate}}

\bigskip
\Recet{Velouté de carottes au curry}{Préparation : 20 minutes \\ Cuisson : 35 minutes}
{Pour 4 personnes}{\begin{itemize}
	\item 1 oignon
	\item 1 courgettes
	\item 600 g de carottes
	\item 1 cuillère à café de curry en poudre
	\item sel,poivre
	\item 2 cuillère à soupe de crème fraîche épaisse
	\item 1 cuillère à soupe d'huile d'olive
\end{itemize}}
{\phantom{.}

\bigskip
\begin{enumerate}
	\item Laver les carottes, les peler puis les couper grossièrement. Laver les courgettes, les peler. Les couper en petits dés. Peler et émincer l'oignon.
	\item Dans une cocotte, faire chauffer l'huile d'olive, y dorer l'oignon émincé puis ajouter les courgettes et les carottes. Saler, poivrer et ajouter le curry en poudre. Couvrir d'eau et laisser cuire sur feu doux 35 minutes environ. (ajouter un peu d'eau si besoin).
	\item En fin de cuisson ajouter la crème fraîche et mixer afin d'obtenir un velouté.
\end{enumerate}

\bigskip
\phantom{.}}

\bigskip
\Recet{Velouté de légumes}{Préparation : 20 minutes \\ Cuisson : 30 minutes}
{}{\begin{itemize}
	\item 2 pommes de terre
	\item 3 carottes
	\item 1 poireaux
	\item 1 navet
	\item 1 oignon 
\end{itemize}}
{\begin{enumerate}
	\item Epluchez, lavez et coupez tous les légumes en petits morceaux.
	\item Mettez-les dans une casserole. Ajoutez de l'eau au ras des légumes. Couvrez et faites cuire à feu moyen pendant 30 minutes.
	\item A la fin de la cuisson, hors du feu, passez la soupe avec un pied à soupe. Salez et poivrez à votre convenance. 
\end{enumerate}

\underline{Astuce :} On peut rajouter un peu sauce Soja !}

\bigskip
\Recet{Velouté de poireaux}{Préparation : 20 minutes \\ Cuisson : 40 minutes}
{Pour 4 personnes}{\begin{itemize}
	\item 4 poireaux
	\item 1 grosse pomme de terre
	\item 1 oignon moyen
	\item 2 cuillères à soupe d'huile d'olive
	\item 1/2 l de bouillon de poule (soit une bouillon cube dilué dans 1/2 l d'eau bouillante) 
\end{itemize}}
{\begin{enumerate}
	\item Détailler la partie blanche des poireaux en petits tronçons et les passer sous l'eau dans une passoire pour bien les laver. 
	\item Faire revenir l'oignon émincé dans une casserole à feu doux, avec l'huile d'olive. Une fois qu'il est translucide, ajouter les poireaux et les faire revenir le temps qu'ils dorent un peu et qu'ils "fondent".
	\item Ajouter la pomme de terre coupée en morceaux, ainsi que le bouillon de poule. Laisser mijoter à feux doux pendant 35 minutes.
	\item Passer le tout au blender ou au presse-purée électrique afin d'obtenir un mélange homgène assez fin. 
\end{enumerate}

\underline{Astuce :} Pour rendre la soupe plus onctueuse, on peut rajouter un peu de crème fraîche ou de lait à la fin !}

\bigskip
\Recet{Velouté de potiron}{Préparation : 30 minutes \\ Cuisson : 45 minutes}
{Pour 4 personnes}{\begin{itemize}
	\item 1 quartier de potiron
	\item 1 gros oignon
	\item 1 brique de crème fraîche liquide
	\item sel, poivre
	\item muscade en poudre
	\item un peu d'huile
\end{itemize}}
{\begin{enumerate}
	\item Couper la chair de la courge en gros dés.
	\item Couper l'oignon en lamelles et le faire revenir dans une cocotte avec un peu d'huile.
	\item Ajouter les dés de courge dans la cocotte et recouvrir d'eau (juste au niveau de la courge, pas plus).
	\item Laisser bouillir environ 45 min à 1 h, quand la fourchette rentre facilement dans un dé de potiron.
	\item Mixer le tout. Verser la crème liquide, saler, poivrer et ajouter de la muscade selon votre goût. 
\end{enumerate}}

\bigskip
\Recet{Soupe Avoine et Shiitakés}{}
{Pour 4 personnes}{\begin{itemize}
	\item 2 cuillerées à soupe d'huile d'olive
	\item 1 échalote
	\item 125g de shiitakés
	\item 1 litre de bouillon de légumes
	\item 75g de flocons d'avoine cuisson rapide (quaker oats) 
	\item Un peu de persil finement ciselé
	\item Sel, poivre
\end{itemize}}
{\begin{enumerate}
	\item Épluchez l'échalote et coupez les champignons en morceaux.
	\item Faites chauffer l'huile d'olive dans une grande casserole et ajoutez les morceaux d'échalote. Peu importe qu'ils soient bien coupés, car tout est mixé dans moins de 10 minutes ! 
	\item Ajoutez les shiitakés. Laissez cuire 2 minutes en remuant pour éviter que le fond ne brûle. Versez le litre de bouillon de légumes.
	\item Portez à ébullition et laissez cuire 2 minutes de plus. Versez les flocons d'avoine en une fois. Mélangez et faites bouillir 2 minutes.
	\item Ajoutez un peu de persil haché, salez et poivrez à votre goût.
	\item Mixez finement pour obtenir un velouté. C'est prêt !
\end{enumerate}}

\bigskip
\Recet{Crème de chou-fleur et maquereau fumé}{Préparation : 10 minutes \\ Cuisson : 20 minutes}
{Pour 4 personnes}{\begin{itemize}
	\item 500g de chou-fleur
	\item 100g de maquereau fumé
	\item 20cl de crème liquide
	\item Sel et poivre du moulin
\end{itemize}}
{\begin{enumerate}
	\item Portez à ébullition une grande casserole d'eau salée. Détachez les fleurettes de chou-fleur, puis lavez-les sous l'eau fraîche. Plongez-y les fleurettes, puis laissez cuire à petits bouillons pendant 20 minutes.
	\item Émiettez le maquereau fumé.
	\item Égouttez les fleurettes de choux, puis mixez-les finement avec la crème liquide et un peu de bouillon de cuisson pour obtenir la consistance d'un velouté. Ajoutez sel et poivre.
	\item Servez-le immédiatement avec les miettes de maquereau fumé.
\end{enumerate}}
